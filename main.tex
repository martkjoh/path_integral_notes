\documentclass[10pt, a4paper, oneside]{book}

\usepackage[
  left=1.5cm, 
  right=1.5cm, 
  top=2cm, 
  bottom=2cm, 
]{geometry}

% math mode additions
\usepackage{amsmath,amsfonts,amssymb,mathtools}

% nice derivatives
\usepackage[ISO]{diffcoeff4}

% in-document references
\usepackage{hyperref}

% Colors, f.ex. in mdframed boxes
\usepackage[dvipsnames]{xcolor}

% make todo-notes
\usepackage[
  textsize=scriptsize,
  textwidth=1.5cm,
  linecolor=blue,
  backgroundcolor=cyan,
  colorinlistoftodos
  ]{todonotes}

% Text boxes
\usepackage{framed}

% Braket notation
\usepackage{braket}

% to write 'goes to zero/infinity' in equatoations
\usepackage{cancel}

% Tabel of contents
\usepackage{tocloft}

% Placing graphics
\usepackage{graphicx,caption,subcaption}

% Better titlepage
\usepackage{titlesec}

% Bold math
\usepackage{bm}

% Feynman diagrams
% \usepackage{feynmf}
\usepackage{feynmp-auto}
\setlength{\unitlength}{1mm}

\usepackage[%
    style=numeric-comp, % Combine consecutive citations, e.g., [15]-[19]
    sorting=none,       % Sorts citations after their appearance in the document
    sortcites=true,     % Sorts within one "autocite", e.g., [15][17][44]
    doi=true,
    giveninits=true,    % use initials
    % issn=false,
    backend=biber,
    hyperref,
    % pages=false
    ]{biblatex}
\DeclareFieldFormat{pages}{#1}



\AtEveryBibitem{} 
\AtEveryBibitem{
    \clearfield{issue} % Zotero messes up number and issue field
    \clearfield{urlyear}
    \clearfield{urlmonth}
	\clearfield{note}
}

% New names for derivatives
\newcommand{\odv}[3][1]{\diff[#1]{#2}{#3}}
\newcommand{\pdv}[3][1]{\diffp[#1]{#2}{#3}}
\newcommand{\fdv}[3][1]{\diff.delta.[#1]{#2}{#3}}

% Shorthands
\newcommand{\dd}{\mathrm{d}}

% Expectation value
\newcommand{\E}[1]{\left \langle #1 \right \rangle}

% Fancy letters     
\newcommand{\Eff}{\mathcal F}
\newcommand{\D}{\mathcal D}
\newcommand{\Ce}{\mathcal C}
\newcommand{\A}{\mathcal A}
\newcommand{\Oh}{\mathcal O}
\newcommand{\V}{\mathcal V}
\newcommand{\Ci}{\mathcal C}
\newcommand{\N}{\mathcal N}
\newcommand{\He}{\mathcal H}
\newcommand{\Ell}{\mathcal L}
\newcommand{\Em}{\mathcal M}
\newcommand{\J}{\mathcal J}
\newcommand{\Es}{\mathcal S}
\newcommand{\Ge}{\mathcal G}
\newcommand{\Te}{\mathcal T}
\newcommand{\Ex}{\mathcal X}
\newcommand{\Pe}{\mathcal P}
\newcommand{\Ess}{\mathcal S}
\newcommand{\Essdot}{\mathcal{\dot S}}
\newcommand{\Eh}{\mathcal E}
\newcommand{\En}{\mathcal N}
\newcommand{\Je}{\mathcal J}
\newcommand{\T}{\mathcal T}
\newcommand{\K}{\mathcal K}
\newcommand{\C}{\mathbb C}
\newcommand{\R}{\mathbb R}
\newcommand{\Z}{\mathbb Z}
\newcommand{\NN}{\mathbb N}

\newcommand{\br}{\mathbf}

\renewcommand{\Im}{\mathrm{Im}}
\renewcommand{\Re}{\mathrm{Re}}


% Operations
\newcommand{\Tr}{\mathrm{Tr}}

% Means-square limit
\newcommand{\mslim}[1]{ \underset{#1}{ \operatorname{ms-lim} }}
\newcommand{\res}{\mathrm{Res}}
\newcommand{\const}{\mathrm{const.}}
\DeclareMathOperator{\sgn}{sgn}

% identity operator
\newcommand{\one}{\text{\usefont{U}{bbold}{m}{n}1}}
\MakeRobust{\one}

% Set builder notation
\newcommand{\setbuilder}[2]{\left\{\, #1 \mid #2 \,\right\}}

% comments
\newcommand{\note}[1]{\todo[inline,size=\normalsize,color=SkyBlue]{ #1 }}
\newcommand{\MJ}[1]{\textcolor{red}{\texttt{[MJ: #1]}}}


% Thanks to Kleinert!
% Make feynmp nice
\newcommand{\setval}{
    \fmfset{wiggly_len}{2 mm}
    \fmfset{arrow_len}{3mm}
    \fmfset{arrow_ang}{13}
    \fmfset{dash_len}{2mm}
    \fmfpen{0.25mm}
    \fmfset{dot_size}{1.5thick}
}

\newcommand{\fmftri}[1]{\fmfv{decor.shape=triangle,decor.angle=-90,decor.size=1.2thick}{#1}}
\newcommand{\fmftriinv}[1]{\fmfv{decor.shape=triangle,decor.angle=90,decor.size=1.2thick}{#1}}
\newcommand{\fmfsq}[1]{\fmfv{decor.shape=square,decor.size=1.2thick}{#1}}
\newcommand{\fmfsqside}[1]{\fmfv{decor.shape=square,decor.angle=45,decor.size=1.2thick}{#1}}
\newcommand{\fmfcounter}[1]{\fmfv{decor.size=2thick,l=${\bm{\otimes}}$,l.dist=0}{#1}}
\newcommand{\comma}{,} % This is needed for commas in the feynman diagrams...

\title{Lecture Notes: Path Integral Methods in Stochastic Processes and Field Theory}
\author{Luca Cocconi, Gennaro Tucci, Martin Johnsrud}
\date{October 2024}

\begin{document}

\maketitle

\chapter{Introduction}
This is the lecture notes from the course ``Path integral methods in stochastic processes and field theory'' held at Göttingen University and the Max Planck Institute of Dynamics and Self-organization

\chapter{Onsager-Machlup path integral}

In this section, we will derive the Onsager-Machlup path integral.
This is a formulation of Markovian stochastic processes which allows for describing the conditional probability of the resulting state physical process, given its initial conditions, as a sum over all possible ways, or ``paths'', that could result in that state.


\section{Stochastic processes}

A stochastic process is $x(t)$ is a function of $t \in \Te$, where the only requirement on $\Te$ is that it is \emph{total ordered}.
This means that, for any two elements $t_1, t_2 \in \Te$ and $t_1 \neq t_2$, then $t_1 < t_2$ or $t_2 < t_1$.
The most common example of $\Te$ is time, so $x(t)$ describes the evolution of $x$ thought time.
However, $\Te$ might also be the links in a polymer, \dots \todo{examples}
Furthermore, $\Te$ can be discrete or continuous.
We will begin by considering discrete time steps with a length of $\Delta t$, so
%
\begin{align}
    \Te = \Delta t \Z.
\end{align}
%

A \emph{Markov process} is a stochastic process with ``no memory''.
This means that, if we have $n$ steps of the process $x(t_1), x(t_2), \dots x(t_n)$, then the conditional probability of $x(t_n)$, given $x(t_1), \dots x(t_{n-1})$, only depends on the last step $x(t_{n-1})$.
Using the common notation for conditional probabilities, where $P(A|B)$ means ``the probability of $A$ given $B$'', this can be stated as
%
\begin{align}
    P(x_n |x_{n-1}, x_{n-2} \dots x_1) = P(x_n | x_{n-1}).
\end{align}
%
Here, we use the shorthand $x_i = x(t_i)$.

As far as we know, the underlying laws of physics are Markovic, so the question of 

\note{I will fill in more text later}

This does not imply statistical independence, so $P(x_{n}, x_{n-1})\neq P(x_n)P(x_n)$


A Markovian process has the property
%
\begin{align}
    P(x_1, x_2, x_3) = P(x_3|x_2)P(x_2|x_1)P(x_1).
\end{align}
%
From this, we may derive the Chapman-Kolmogorov equation,\todo{detail?}
%
\begin{align}
    P(x_3|x_1) = \sum_{x_2} P(x_3|x_2) P(x_2|x_1).
\end{align}
%
Here, the sum is over all possible values of $x_2$.
\note{Something like: we consider $x$ discrete, $x \in \Delta x \Z $. }

Repeatedly applying this gives the conditional probability of two steps $x_1$ and $x_{n+1}$ arbitrarily far removed, 
%
\begin{align}\label{eq: cond prob markov x0 given xn}
    P(x_{n+1}|x_0) 
    = \sum_{\{ x_1, \dots x_n \}}
    P(x_{n+1}|x_n) P(x_n| x_{n-1})\dots P(x_2|x_1).
\end{align}
%
We see that this already begins to resemble something like a sum over all possible paths.

\note{more text}

\note{Illustration}

We will now take two different limits, which will yield the path integral.
This means we will have to take the continuum limit.

\note{Is this the right way to do it?}
To do this, we must consider probability \emph{densities}, $\Pe(x) = P(x) / \Delta x $.
The densities are ``probability of the value $x$ per $\Delta x$''.
This allows us to go from sums to integrals,
%
\begin{align}
    \sum_{x} P(x) = \sum_{x} \Delta x \Pe(x) \underset{\Delta x \rightarrow 0}{\longrightarrow} \int \dd x \, \Pe(x).
\end{align}
%

\note{Define $\D x(t)$}


\section{Gaussian process and the Ornstein-Uhlenbeck process}

The most important class of stochastic processes, as it is pretty much the only one we can solve, is Gaussian processes.
These have the form
%
\begin{align}
    P(x_{n_1}|x_n)
    = 
    \frac{ 1 }{ \sqrt{ 2 \pi \det \Sigma } }
    \exp \left\{ - (x_{n+1}  - \bar x_{n+1}) \Sigma^{-1} (x_{n+1}  - \bar x_{n+1}) \right\}.
\end{align}
%
Here, $\bar x_{n+1}$ is the expected value of $x_n$.
As we consider Markovian processes, this is a function of only the previous step, $x_n$
$\Sigma$ is the covariance of the process.

\todo{Why matrix notation?}

A specific example is the Ornstein-Uhlenbeck process,
%
\begin{align}
    \odv{ x(t) }{ t } = - \mu x(t) + \eta(t).
\end{align}
%

\note{details \dots}

This has the conditional probability
%
\begin{align}
    P(x_{n + 1}| x_n) 
    = \sqrt{ \frac{ \mu }{ 4 \pi D \Delta t } }
    \exp \left\{ 
    \frac{ \left(x_{n + 1} - x_n + \mu x_n \Delta t\right)^2 }{ 4 D \Delta t } 
    \right\}
    = \sqrt{ \frac{ \mu }{ 4 \pi D \Delta t } }
    \exp \left\{ 
    \frac{ \Delta t }{ 4 D }  \left(\dot x_{n + 1} + \mu x_n\right)^2
    \right\},
\end{align}
%
where we have introduced the discrete time-derivative, $\dot x_{n+1} = (x_{n + 1} - x_n) / \Delta t$.
If we insert this into \autoref{eq: cond prob markov x0 given xn}, we get
%
\begin{align}
    P(x_{n + 1} | x_1) 
    = \int \D x(t) \,
    \exp \left\{ 
        - \frac{ 1 }{ 4D } 
        \int\limits_{t_1}^{t_{n+1}} \dd t \left[\dot x(t) + \mu x(t)\right]^2
        \right\}
    \equiv
    \int \D x(t) \, \Phi[x(t)].
\end{align}
%
$\Phi$ is the Onsager-Machlup \emph{functional}.
It is a functional, and not a function, as it takes in a function $x(t)$, and gives back a number.

\section{Master equation}

\note{Fill}

\section{Gaussian integrals}

\note{Fill}



\end{document}
