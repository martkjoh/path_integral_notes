The Ornstein-Uhlenbeck process, \autoref{eq: OUP} is an example of a Langevin equation, a differential equation where one of the terms, $\eta$ in that case, is a stochastic variable.
To treat these forms of equations we must generalize the well familiar rules of calculus to \emph{stochastic calculus}.
This chapter will serve as a short primer on stochastic calculus.
For a modern introduction, see for example~\footnote{T. A. de Pirey, L. F. Cugliandolo, V. Lecomte, and F. van Wijland, Path integrals and stochastic calculus, Advances in Physics 71, 1 (2022). \href{https://doi.org/10.48550/arXiv.2211.09470}{ arXiv:2211.09470v2 }
}~\footnote{
    C. Gardiner, Stochastic Methods: A Handbook for the Natural and Social Sciences (Springer Berlin Heidelberg, 2009).
}


\section{Langevin equations}

We consider Langevin equations of the form
%
\begin{align}
    \odv{  }{ t } x(t) = f(x(t)) + g(x(t)) \eta(t).
\end{align}
%
Here, $f(x(t))$ is the \emph{drift}-term, which often can be interpreted as deterministic forces, $g(x(t))$ is the \emph{noise amplitude}, while $\eta(t)$ is \emph{Gaussian white noise}.
This is a stochastic variable, with the following statistical properties,
%
\begin{align}
    \E{\eta(t)} &= 0, &
    \E{\eta(t)\eta(t')} &= \delta(t - t').
\end{align}
%
The ``delta-correlation'' means that the noise have no memory, there is no correlation between the value at different times.
This Langevin equation is easily generalized to several degrees of freedom by promoting $x$, $f$ and $\eta$ as vectors, and $g$ as a matrix.

If the amplitude of the noise is independent of the value of $x$, so $g = \const$, we say the equation has \emph{additive noise}.
If, instead, $g = g(x)$ is non-constant, we say equation has \emph{multiplicative noise}.
Additive noise is a lot easier to deal with that multiplicative noise, as we will see.

We list some examples of Langevin equations, and where they appear:
\begin{itemize}
    \item \textit{Brownian motion}: The simplest case, where $f = 0$ and $g = \sqrt{ 2 D }  = \const $ yields 
    %
    \begin{align}
        \odv{  }{ t } W(t)  = \eta(t).
    \end{align}
    %
    $W(t)$ is called a \emph{Wiener-process}, and models Brownian motion such as pollen floating in water.
    \item \emph{The Ornstein-Uhlenbeck process}: We encountered the OUP earlier, in \autoref{sectoin: gaussian and OUP}.
    In its original formulation, the variable $v(t)$ modeled the velocity of a particle, $f(v) = - \mu v$ is the friction force.
    The noise amplitude $g = \sqrt{ 2 \mu k_B T }$ is then given by the Einstein-relation, and Newton's second law takes the form of a Langevin equation:
    %
    \begin{align}
        m \odv{  }{ t } v(t) = - \mu v(t) + \sqrt{ 2 \mu k_B T } \eta(t).
    \end{align}
    %
    More generally, processes with a linear restoring force $f(x) = c x$ and additive noise are called Ornstein-Uhlenbeck processes.
    \item \emph{The Black–Scholes equation}: 
    This models the price of an \emph{option} over time, $s(t)$, a financial instrument all analogous to a stock.
    This is modeled with a linear drift, as in the case of the OUP, but the fluctuations in price are assumed to increase with the price, so the noise amplitude proportional to $s$ giving multiplicative noise.
    The reuslting Langevin equation is
    %
    \begin{align}
        \odv{}{t} s(t) = \mu s(t) + \sigma s(t) \eta(t).
    \end{align}
    %
    \item \textit{Population models}: The population number $p(t)$ of, for example, a species will for small numbers, grow linearly with size.
    However, an ecosystem will usually have finite res courses, leading to a maximum value $p_\mathrm{max}$ for which the population will start to decline.
    In addition, the fluctuation will depend on the population number---an extinct species cannot random fluctuation back into existance.
    This can be modeled with the following Langevin equation,
    %
    \begin{align}
        \odv{  }{ t } p(t) = \mu (p_\mathrm{max} - p(t)) p(t) + \sigma p(t) \eta(t).
    \end{align}
    %
\end{itemize}

The examples above are all stochastic \emph{ordinary} differential equations (SDE's).
This means the variables are only dependent on one parameter, usually time.
However, stochastic calculus can also be used to extend multivariable calculus, yielding stochastic \emph{partial} differential equations (SPDE's).
We will only list one example here

\begin{itemize}
    \item \emph{Dean's equation}:
    If we consider a set of interaction point particles, $x_\alpha(t)$, obeying the SDE's, we can write down a SPDE for the density function.
    The density function is $\rho(x, t) = \sum_\alpha \delta(x - x_\alpha(t))$, and the resulting evolution equation is
    %
    \begin{align}
        \pdv{  }{ t } \rho(x, t) = \nabla \cdot \left( \rho \nabla \fdv{ F }{ \rho(x,t) } \right) + \nabla \cdot \sqrt{\rho(x, t)} \nabla(x, t).
    \end{align}
    %
    Here, the free-energy functional is
    %
    \begin{align}
        F[\rho] = \int \dd x \rho(x, t) \left[V(x) + \nabla \ln \rho(x, t)\right],
    \end{align}
    %
    where $V$ is the interaction potential.
    The white noise in this Langevin field-equation is delta-correlated in both field and space,
    %
    \begin{align}
        \E{\eta(x, t) \eta(x', t')} = \delta(x - x') \delta(t - t').
    \end{align}
    %
\end{itemize}

One problem with Langevin equations are that it alone is not well-defined, and can rather be called a ``pre-equation''.
Additional info is needed to uniquely define the process that solves the equation.


\section{Time discretization of stochastic processes}

\todo[inline]{Maybe we should write more on why $\Delta \eta\sim \sqrt{ \Delta t }$.}

If we consider a discrete time-step, then we are free to choose if want to evaluate the forces affecting $x$ at the time $t$, $t + \Delta t$, or anywhere between.
In general, the corresponding step in $x$ is
%
\begin{align}
    \Delta x(t) & \equiv x(t + \Delta t) - \Delta x(t)
    =
    f\left(x(t) + \alpha \Delta x(t)\right) \Delta t
    + g\left(x(t) + \alpha \Delta x(t)\right) \Delta \eta,
\end{align}
%
where $\alpha \in [0, 1]$.
For $\alpha \neq 0$, this is an implicit equation for $\Delta x(t)$.

In standard calculus, $x(t)$ is given by the Riemann integral, in which case the choice of discretization $\alpha$ does not affect the result.
However, we may see that this is not the case in stochastic calculus, by expanding the functions for different discretization $\alpha$ and $\alpha'$:
%
\begin{align}
    f\left(x(t) + \alpha \Delta x(t)\right) \Delta t
    - f\left(x(t) + \alpha' \Delta x(t)\right) \Delta t
    & = 
    f'(X(t))(\alpha - \alpha') \Delta x \Delta t = \Oh(\Delta t^{3/2})\\
    g\left(x(t) + \alpha \Delta x(t)\right) \eta t
    - g\left(x(t) + \alpha' \Delta x(t)\right) \eta t
    & = 
    g'(X(t))(\alpha - \alpha') \Delta x \Delta \eta = \Oh(\Delta t).
\end{align}
%
A $\Delta t^{3/2}$-term is sub-leading, so the discretization does not affect this term.
However, in the noise-amplitude, shifting the discretization leads to a $\Oh(\Delta t)$ term, which will affect the result.
The choice of discretization is thus only relevant when dealing with multiplicative noise.
But in this case, a Langevin equation is only well-defined if we specify the discretization.

We therefore add the discretization above the equality in Langevin equation:
%
\begin{align}
    \odv{  }{ t } x(t) \overset{\alpha}{=} f(x(t)) = g(x(t)) \eta(t).
\end{align}
%
This is now a well-defined equation!

The most common choices of discretization and their names are
%
\begin{align}
    \alpha
    =
    \begin{cases}
        0, & \text{Itô}, \\
        \frac{1}{2}, & \mathrm{Stratonovich}, \\
        1, & \text{Hänggi-Klimontovich/Anti-Itô},
    \end{cases}
\end{align}
%
The ``right'' choice of $\alpha$ often depends on the source of the equation. 
If one starts with an equation with a finite correlation time and inertia,
%
\begin{align}
    m \odv[2]{   }{ t } x(t) + \mu \odv{  }{ t } x(t) + g(x(t)) \eta(t) = 0,
\end{align}
%
where\todo{is this right?}
%
\begin{align}
    \E{\eta(t)\eta(t')} = \frac{1}{\tau} e^{-|t - t'|^2 / \tau}
\end{align}
%
then the correct discretization in the overdamped ($m\rightarrow 0$) and short correlation time ($\tau \rightarrow 0$) limits yields a white-noise, first order Langevin equation, but with different discretization depending on the order of the limit.

The choice of discretization may also be one of convenience.
Given a Langevin equation in one discretization, one find a \emph{different} Langevin equation with a \emph{different} discretization, but which yields the same process $x(t)$.
The different discretization have different advantages and drawbacks.
As we saw, Stratonovich yields implicit equations, while Itô yields a different chain-rule, as we will see shortly.


\subsection*{Some useful properties}

To change between discretization, we use the following formula
%
\begin{align}
    \odv{  }{ t } x(t)
    \overset{\alpha}{=} f(x(t)) + g(x(t)) \eta(t)
    \overset{\alpha'}{=} f(x(t)) + (\alpha-\alpha') g'(x(t)) g(x(t)) +  g(x(t)) \eta(t).
\end{align}
%
The additional term for the $\alpha'$-discretization is called a \emph{sporious dift term}.

If we have a new variable, $u(t)$, which is a function of a stochastic process, $u = U(x)$, then we may derive the Langevin equation for this new variable using the stochastic chain rule,
%
\begin{align}
    \odv{  }{ t }u(t) = U'(x(t)) f(x(t)) + U'(x(t)) g\eta + \left(\frac{ 1 }{ 2 } - \alpha \right) U''(x(t)) g^2(x(t)).
\end{align}
%
In the case of Stratonovich, the last term cancels, and we are left with the familiar chain-rule.
In the case of Itô, this is called \emph{Itô's Lemma}.

Lastly, if we want to get rid of multiplicative noise, this is possible for a scalar process, $x \in \R$.
This is done by introducing the following variable process,
%
\begin{align}
    U(x) = \exp \left\{ \int^x \dd x' \frac{1}{g(x)} \right\}.
\end{align}
%
In the noise term, $U'(x) g(x) \eta(t)$, the change-of-variable factor $U'$ will then cancel the noise-strength $g(x)$.
Some care is necessary here, to make sure this transformation is well-defined.
This is not generally possible in higher dimensions.
