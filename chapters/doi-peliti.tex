The response and Onsager-Machlup formalism is useful for perturbative calculations, however, it relies on some assumption which do not always hold.
Most importantly, if we cannot describe the small time-step as a Gaussian process, or/and the random variable in question cannot be modeled as continuous, then it fails.
Examples of discrete processes are birth and death processes with low population number, or discrete jump processes, automata or reaction-diffusion processes.
In this case, we can instead use the Doi-Peliti formalism.
To derive this, we start with the master equation, as we derived in (REF).

\section{Discrete master equation}

We will consider one or more random variables $N(t) \in \NN$.
The corresponding master equation is
%
\begin{align}
    \partial_t P(N, t)
    = 
    \sum_{N'}
    \left\{
        W_t(N | N') P(N, t)
        - W_t(N'|N)P(N, t)
    \right\}.
\end{align}
%
To recap, the first term is the gain term, while the second is the loss-term.
The $W$'s are transition rates, infinite dimensional matrices that contain the rates at which one state transition into another.
The result is an infinite set of ODE's for the probabilities $P(N, t)$.
Some examples of these processes are
\note{It is a bit wierd to have partial derivative, then call it ode's. Maybe should write $P_N(t)$, then ordianry derivative instead?}

\subsection*{Extinction
$
\left(
    \parbox{20mm}{
    \centering
    \begin{fmfgraph*}(8,2)
        \setval
        \fmfleft{i}
        \fmfright{o}
        \fmf{fermion,label=$\epsilon$}{i,o}
        \fmflabel{$A$}{i}
        \fmflabel{$\emptyset$}{o}
    \end{fmfgraph*}
    }
\right)
$
}
This process represents one set of particles, labeled $A$, which decays at a constant rate, $\epsilon$, to nothing $(\emptyset)$, as illustrated by the arrow above.
$N(t)$ then represent the total numbers of particles of type $A$,
and the transition rates are the decay rate, $\epsilon$, times the number of particles.
The corresponding master equation is represented the rate at which the system transitions into, minus the rat out of, state $N$,
%
\begin{align}
    \partial_t P(N, t) = 
    \epsilon
    \left[
        (N + 1) P(N+1, t)
        - 
        N P(N, t)
    \right].
\end{align}
%
The steady-state is a probability distribution $P(N, t) = P_S(N, t$ ) so that $\partial_t P(N) = 0$, and the only such state is
%
\begin{align}
    P_S(N) = \delta_{N,0},
\end{align}
%
the ``emtpy state''.

We may derive a differential equation for the expected number of particles, $\E{N(t)}$, by ``applying'' $\sum_N N \cdot$ to  the master equation, which yields
%
\begin{align}
    \partial_t
    \underbrace{\sum_N N P(N, t)}_{\E{N(t)}}
    &= 
    \epsilon
    \sum_{N}
    \bigg\{
        \underbrace{N(N + 1)}_{(N_1)^2-(N+1)}P(N+1,t)
        - NP(N,t) k
    \bigg\}\\
    & = 
    \epsilon
    \left\{
        \E{N(t)^2} - \E{N(t)} - \E{N(t)^2}
    \right\}
    =     \epsilon \E{N(t)},
\end{align}
%
This is solved by $\E{N(t)} = N(0) e^{-\epsilon t}$.


\subsection*{Spontaneous creation
$
\left(
    \parbox{20mm}{
    \centering
    \begin{fmfgraph*}(8,2)
        \setval
        \fmfleft{i}
        \fmfright{o}
        \fmf{fermion,label=$\gamma$}{i,o}
        \fmflabel{$\emptyset$}{i}
        \fmflabel{$A$}{o}
    \end{fmfgraph*}
    }
\right)
$
}

This is the converse process, with the master equation
%
\begin{align}
    \partial_t P(N, t)
    = 
    \gamma 
    \left\{
        P(N - 1, t) - P(N, t)
    \right\}.
\end{align}
%
One may notice this has no steady state, as this would require $P_S(N) = P_S(N+1)$, a uniform distribution on the natural numbers, which is unnormalizable.


\subsection*{Dynamic steady state
$
\left(
    \parbox{34mm}{
    \centering
    \begin{fmfgraph*}(22,2)
        \setval
        \fmfleft{i}
        \fmfright{o}
        \fmf{fermion,label=$\gamma$}{i,c1}
        \fmf{phantom,tension=2}{c1,c2}
        \fmf{fermion,label=$\epsilon$}{c2,o}
        \fmflabel{$\emptyset$}{i}
        \fmflabel{$B$}{o}
        \fmfv{l=$A$,l.a=0,l.d=.03w}{c1}
    \end{fmfgraph*}
    }
\right)
$
}

A generic property of the processes considered here is that we may always consider the combination of two processes by adding together the right-hand sides of the corresponding master equations.
The master equation for the spontaneous creation and annihilation process is therefore
%
\begin{align}
    \partial_t P(N, t) = 
    \epsilon
    \left[
        (N + 1) P(N+1, t)
        - 
        N P(N, t)
    \right]
    +
    \gamma 
    \left\{
        P(N - 1, t) - P(N, t)
    \right\}.
\end{align}
%
Assuming $P(N<0) = 0$ gives the steady state,
%
\begin{align}
    P_S(N) = \frac{ P_S(0) }{ N! } \left(\frac{ \gamma }{ \epsilon }\right)^N
    =
    \frac{ 1 }{ N! } \left(\frac{ \gamma }{ \epsilon }\right)^N e^{-\gamma/\epsilon},
\end{align}
%
which is the Poission distribution.

A couple of other processes, and their rates, are

\subsection*{Coagulation
$
\left(\,
    \parbox{22mm}{
    \centering
    \begin{fmfgraph*}(8,2)
        \setval
        \fmfleft{i}
        \fmfright{o}
        \fmf{fermion,label=$\lambda$}{i,o}
        \fmflabel{$2A$}{i}
        \fmflabel{$A$}{o}
    \end{fmfgraph*}
    }
\right)
$
}

%
\begin{align}
    W(N-1|N) = \lambda N 2
\end{align}
%


\subsection*{Transmutation
$
\left(
    \parbox{20mm}{
    \centering
    \begin{fmfgraph*}(8,2)
        \setval
        \fmfleft{i}
        \fmfright{o}
        \fmf{fermion,label=$\gamma$}{i,o}
        \fmflabel{$A$}{i}
        \fmflabel{$B$}{o}
    \end{fmfgraph*}
    }
\right)
$
}

%
\begin{align}
    W(N_A-1,N_B+1|N_A,N_B) 
    = \tau N_A
\end{align}
%

\subsection*{Spawning
$
\left(
    \parbox{30mm}{
    \centering
    \begin{fmfgraph*}(22,6)
        \setval
        \fmfleft{i,i2}
        \fmfright{o,o2}
        \fmf{fermion}{i2,c,o2}
        \fmflabel{$s$}{c}
        \fmffreeze
        \fmf{fermion}{c,o}
        \fmflabel{$A$}{i2}
        \fmflabel{$A$}{o}
        \fmflabel{$B$}{o2}
    \end{fmfgraph*}
    }
\right)
$
}

%
\begin{align}
    W(N_A,N_B+1|N_A,N_B) 
    = s N_A
\end{align}
%


\subsection*{Catalysis
$
\left(
    \parbox{30mm}{
    \centering
    \begin{fmfgraph*}(22,6)
        \setval
        \fmfleft{i,i2}
        \fmfright{o,o2}
        \fmf{fermion}{i2,c,o2}
        \fmflabel{$t$}{c}
        \fmffreeze
        \fmf{fermion,left=1/2}{i,c,o}
        \fmflabel{$A$}{i2}
        \fmflabel{$B$}{o2}
        \fmflabel{$C$}{i1}
        \fmflabel{$C$}{i2}
    \end{fmfgraph*}
    }
\right)
$
}

%
\begin{align}
    W(N_A-1,N_B+1, N_C|N_A,N_B, N_C) 
    = t N_A N_C
\end{align}
%