The response and Onsager-Machlup formalism is useful for perturbative calculations, however, it relies on some assumption which do not always hold.
Most importantly, if we cannot describe the small time-step as a Gaussian process, or/and the random variable in question cannot be modeled as continuous, then it fails.
Examples of discrete processes are birth and death processes with low population number, or discrete jump processes, automata or reaction-diffusion processes.
In this case, we can instead use the Doi-Peliti formalism.
To derive this, we start with the master equation, as we derived in \autoref{section: master equation}.

\section{Discrete master equation}

We will consider one or more random variables $N(t) \in \NN$.
The corresponding master equation is
%
\begin{align}
    \partial_t P(N, t)
    = 
    \sum_{N'}
    \left\{
        W_t(N | N') P(N, t)
        - W_t(N'|N)P(N, t)
    \right\}.
\end{align}
%
To recap, the first term is the gain term, while the second is the loss-term.
The $W$'s are transition rates, infinite dimensional matrices that contain the rates at which one state transition into another.
The result is an infinite set of ODE's for the probabilities $P(N, t)$.
Some examples of these processes are
\note{Maybe should write $P_N(t)$, then ordianry derivative instead, since its ODEs? LC: sure, sounds good!}

\subsection*{Extinction
$
\left(
    \parbox{20mm}{
    \centering
    \begin{fmfgraph*}(8,2)
        \setval
        \fmfleft{i}
        \fmfright{o}
        \fmf{fermion,label=$\epsilon$}{i,o}
        \fmflabel{$A$}{i}
        \fmflabel{$\emptyset$}{o}
    \end{fmfgraph*}
    }
\right)
$
}
This process represents one set of particles, labeled $A$, which decays at a constant rate, $\epsilon$, to nothing $(\emptyset)$, as illustrated by the arrow above.
$N(t)$ then represent the total numbers of particles of type $A$,
and the transition rates are the decay rate, $\epsilon$, times the number of particles.
The corresponding master equation is represented the rate at which the system transitions into, minus the rate out of, state $N$,
%
\begin{align}
    \partial_t P(N, t) = 
    \epsilon
    \left[
        (N + 1) P(N+1, t)
        - 
        N P(N, t)
    \right].
\end{align}
%
The steady-state is a probability distribution $P(N, t) = P_S(N, t)$ so that $\partial_t P(N) = 0$, and the only such state is
%
\begin{align}
    P_S(N) = \delta_{N,0},
\end{align}
%
the ``emtpy state''.

We may derive a differential equation for the expected number of particles, $\E{N(t)}$, by ``applying'' $\sum_N N \cdot$ to  the master equation, which yields
%
\begin{align}
    \partial_t
    \underbrace{\sum_N N P(N, t)}_{\E{N(t)}}
    &= 
    \epsilon
    \sum_{N}
    \bigg\{
        \underbrace{N(N + 1)}_{(N_1)^2-(N+1)}P(N+1,t)
        - NP(N,t) k
    \bigg\}\\
    & = 
    \epsilon
    \left\{
        \E{N(t)^2} - \E{N(t)} - \E{N(t)^2}
    \right\}
    =     \epsilon \E{N(t)},
\end{align}
%
This is solved by $\E{N(t)} = N(0) e^{-\epsilon t}$.


\subsection*{Spontaneous creation
$
\left(
    \parbox{20mm}{
    \centering
    \begin{fmfgraph*}(8,2)
        \setval
        \fmfleft{i}
        \fmfright{o}
        \fmf{fermion,label=$\gamma$}{i,o}
        \fmflabel{$\emptyset$}{i}
        \fmflabel{$A$}{o}
    \end{fmfgraph*}
    }
\right)
$
}

This is the converse process, where particles of species A pop into existence at a constant rate $\gamma$ independent of the current occupation, with the master equation
%
\begin{align}
    \partial_t P(N, t)
    = 
    \gamma 
    \left\{
        P(N - 1, t) - P(N, t)
    \right\}.
\end{align}
%
One may notice this has no steady state, as this would require $P_S(N) = P_S(N+1)$, a uniform distribution on the natural numbers, which is unnormalizable.


\subsection*{Dynamic steady state
$
\left(
    \parbox{34mm}{
    \centering
    \begin{fmfgraph*}(22,2)
        \setval
        \fmfleft{i}
        \fmfright{o}
        \fmf{fermion,label=$\gamma$}{i,c1}
        \fmf{phantom,tension=2}{c1,c2}
        \fmf{fermion,label=$\epsilon$}{c2,o}
        \fmflabel{$\emptyset$}{i}
        \fmflabel{$B$}{o}
        \fmfv{l=$A$,l.a=0,l.d=.03w}{c1}
    \end{fmfgraph*}
    }
\right)
$
}

A generic property of the processes considered here is that we may always consider the combination of two (concurrent) processes by adding together the right-hand sides of the corresponding master equations.
The master equation for the spontaneous creation and annihilation process is therefore
%
\begin{align}
    \partial_t P(N, t) = 
    \epsilon
    \left[
        (N + 1) P(N+1, t)
        - 
        N P(N, t)
    \right]
    +
    \gamma 
    \left\{
        P(N - 1, t) - P(N, t)
    \right\}.
\end{align}
%
Assuming $P(N<0) = 0$ gives the steady state,
%
\begin{align}
    P_S(N) = \frac{ P_S(0) }{ N! } \left(\frac{ \gamma }{ \epsilon }\right)^N
    =
    \frac{ 1 }{ N! } \left(\frac{ \gamma }{ \epsilon }\right)^N e^{-\gamma/\epsilon},
\end{align}
%
which is the Poission distribution. This statistical steady-state may be defined as ``dynamic'' since the occupation number for a given realisation of the process may still vary over time.\\

A couple of other processes, and their rates, are

\subsection*{Coagulation
$
\left(\,
    \parbox{22mm}{
    \centering
    \begin{fmfgraph*}(8,2)
        \setval
        \fmfleft{i}
        \fmfright{o}
        \fmf{fermion,label=$\lambda$}{i,o}
        \fmflabel{$2A$}{i}
        \fmflabel{$A$}{o}
    \end{fmfgraph*}
    }
\right)
$
}

%
\begin{align}
    W(N-1|N) = \lambda \binom{N}{2}
\end{align}
%


\subsection*{Transmutation
$
\left(
    \parbox{20mm}{
    \centering
    \begin{fmfgraph*}(8,2)
        \setval
        \fmfleft{i}
        \fmfright{o}
        \fmf{fermion,label=$\gamma$}{i,o}
        \fmflabel{$A$}{i}
        \fmflabel{$B$}{o}
    \end{fmfgraph*}
    }
\right)
$
}

%
\begin{align}
    W(N_A-1,N_B+1|N_A,N_B) 
    = \tau N_A
\end{align}
%

\subsection*{Spawning
$
\left(
    \parbox{30mm}{
    \centering
    \begin{fmfgraph*}(22,6)
        \setval
        \fmfleft{i,i2}
        \fmfright{o,o2}
        \fmf{fermion}{i2,c,o2}
        \fmflabel{$s$}{c}
        \fmffreeze
        \fmf{fermion}{c,o}
        \fmflabel{$A$}{i2}
        \fmflabel{$A$}{o}
        \fmflabel{$B$}{o2}
    \end{fmfgraph*}
    }
\right)
$
}

%
\begin{align}
    W(N_A,N_B+1|N_A,N_B) 
    = s N_A
\end{align}
%


\subsection*{Catalysis
$
\left(
    \parbox{34mm}{
    \centering
    \begin{fmfgraph*}(28,6)
        \setval
        \fmfleft{i,i2}
        \fmfright{o,o2}
        \fmf{fermion}{i2,c,o2}
        \fmflabel{$t$}{c}
        \fmffreeze
        \fmf{fermion,left=.2}{i,c,o}
        \fmflabel{$A$}{i2}
        \fmflabel{$B$}{o2}
        \fmflabel{$C$}{i}
        \fmflabel{$C$}{o}
    \end{fmfgraph*}
    }
\right)
$
}

%
\begin{align}
    W(N_A-1,N_B+1, N_C|N_A,N_B, N_C) 
    = t N_A N_C
\end{align}
%

\subsection*{General process}

A general process has $k$ species, labeled $A_i$, with $i\in\{1, \dots k\}$, $j_i$ label the number of species $i$ that is needed for the process, while $\ell_i$ are the number of $A_i$ that are output by the reaction (reactants and products), which has rate $r$.
The corresponding diagram is
%
\begin{align}
    \parbox{30mm}{
    \centering
    \begin{fmfgraph*}(22,15)
        \setval
        \fmfleft{i0,i1,i2,i3}
        \fmfright{o0,o1,o2,o3}
        \fmf{fermion}{i0,c,o0}
        \fmf{phantom}{i1,c,o1}
        \fmf{fermion}{i2,c,o2}
        \fmf{fermion}{i3,c,o3}
        \fmfv{d.f=empty,d.shape=circle,d.size=5mm,l=$r$,l.d=0cm}{c}
        \fmflabel{$j_kA_k$}{i0}
        \fmflabel{$\vdots$}{i1}
        \fmflabel{$j_2A_2$}{i2}
        \fmflabel{$j_1A_1$}{i3}
        \fmflabel{$\ell_kA_k$}{o0}
        \fmflabel{$\vdots$}{o1}
        \fmflabel{$\ell_2A_2$}{o2}
        \fmflabel{$\ell_1A_1$}{o3}
    \end{fmfgraph*}
    }
\end{align}
%
and the transition rates are
%
\begin{align}
    W(N_1-\ell_1 + j_1, \dots N_k - \ell_k + J_k| N_1, \dots N_k)
    = r \binom{N_1}{\ell_1} \cdots \binom{N_k}{\ell_k}
\end{align}
%


\subsection*{Processes in space}

We can immediately generalise to processes in space, by considering $A_i$ to be a particle on lattice site $i$ and $N_i(t)$ the instantaneous occupation of said site.
If we consider an equal probability of jumping left or right, hopping rate $D / h^2$, where $h$ is the lattice spacing, then the master equation is
%
\begin{align}
    \partial_t P(\bm N, t) 
    = 
    \frac{ D }{ h^2 } \sum_{j\in \Z}\sum_{\E{x,j}}
    \left[ (N_x + 1) P(\bm N + 1_x - 1_j,t) - N_x P(\bm N, t) \right].
\end{align}
%
Here, $\bm N$ is the vector of occupation number at the different sites, and $1_x$ is the vector with only $1$ at site $x$, $0$ elsewhere, and the bracket means sum over nearest neighbors.
Applying $\sum_{\bm N} N_x \cdot$, we get
%
\begin{align}
    \partial_t \E{N_x(t)} = D \nabla^2_x \E{N_x(t)},
\end{align}
%
the (discrete) diffusion equation. In this sense, the familiar diffusion equation captures only some of the information contained in the original master equation, namely that about average occupation.


\section{Probability generating function}

We define the function
%
\begin{align}
    \Em(z, t) = \sum_N z^N P(N, t)
\end{align}
%
for a single particle species, or for many species,
%
\begin{align}
    \Em(z_1, ... z_K, t) 
    =
    \Em(\bm z, t) 
    = \sum_{\bm N} z^{N_1}\cdots z^{N_k} P(\bm N, t).
\end{align}
%
This is called the probability generating function (PGF) because we can obtain the probability of having the state $N = \ell$ by taking derivatives of it evaluated at $z=0$,
%
\begin{align}
    P(\ell, t) = \frac{ 1 }{ \ell! } \partial_z^\ell \Em(z, t)\big|_{z = 0}.
\end{align}
%
Notice that, for all $t$, $M(z = 1, t) = 1$, which is an important property stemming from normalisation of $P(N,t)$.

If we again consider the master equation for the birth and death process,
%
\begin{align}
    \partial_t P(N, t) = 
    \epsilon
    \left[
        (N + 1) P(N+1, t)
        - 
        N P(N, t)
    \right]
    +
    \gamma 
    \left\{
        P(N - 1, t) - P(N, t)
    \right\},
\end{align}
%
and apply $\sum_N z^N \cdot$, we get a single PDE for $\Em$,
%
\begin{align}
    \partial_t \Em(z, t) 
    &= 
    \epsilon 
    \sum_N 
    \left\{
        \underbrace{z^N (N + 1)}_{\partial_z z^{N + 1}} P(N + 1, t)
        - 
        \underbrace{z^N N}_{z \partial_z z^N} P(N, t )
    \right\}
    +
    \gamma \sum_N
    z^N
    \left\{
         P(N-1, t) - P(N, t)
    \right\}\\
    & = \epsilon (1 - z) \partial_z \Em(z, t)
    + \gamma (z - 1) \Em(z, t).
\end{align}
%
If we consider steady state, $\partial_t \Em = 0$, we get 
%
\begin{align}
    \Em_S(z) = \Em_0 e^{\gamma /\epsilon z}
    \implies
    P_S(N) = 
    \frac{ 1 }{ N! }
    \left(\frac{ \gamma }{ \epsilon }\right)^N 
    e^{-\gamma/\epsilon},
\end{align}
%
as before.

\begin{framed}
    \textit{Exercise:}
    Show that, for the generic process described above, the equation of the generating function is
    %
    \begin{align}
        \partial_t \Em(\bm z, t)
        = \frac{ r }{ \prod_k \ell_k ! }
        \left[
            \left(
                \prod_k z^{j_k}_k \partial_{z_k}^{\ell_k}
            \right)
            -
            \left(
                \prod_k z^{\ell_k}_k \partial_{z_k}^{\ell_k}
            \right)
        \right]
        \Em(\bm z, t).
    \end{align}
    %
\end{framed}

We end this section with a note: although by definition the generating function is analytic in $z$, as it is a power series, the result differential equations may have non-analytic solutions, which must be discarded.
We may also obtain moments from the generating functions by exploiting its analytical properties to write
%
\begin{align}
    \E{N^k(t)}
    = 
    \sum_N \left(z \odv{  }{ z }\right)^k z^N P(N) \big|_{z=1}
    = 
    \left. \left(z \odv{  }{ z }\right)^k \Em(z, t) \right|_{z=1}
\end{align}
Note that the derivatives are evaluate at $z=1$, not $z=0$ as before.



\section{Second quantization}

With the definition and properties of the probability generating function clear, we now introduce a new formalism called \emph{second quantization}, due to its close relation to the similarly named formalism of quantum field theory.
This formalism is a mapping of the objects from the previous chapter to a new notation (first suggested by Masao Doi in 1976).
This mapping has the following dictionary,

\begin{table}[h]
    \centering
    \begin{tabular}{c|c}
        Usual & 2nd quant. \\[.1cm]
        \hline\\[-.2cm]
        Factors of $z^N$ & particle state $\ket{N}$ \\[.1cm]
        Multiplication with $z$ & Creation operator $a^\dagger$\\[.1cm]
        Derivative $\odv{  }{ z }$ & Annihilation operator $a$ \\[.1cm]
        PGF $\Em(z, t)$ & System state $\ket{\Em(t)}$
    \end{tabular}
\end{table}

If the PGF is $\Em(z, t) = z^M$, then $P(N, t) = \delta_{N,M}$ and $N(t)=M$ with probability one. For a generic statistical superposition of occupations, we may write
%
\begin{align}
    \ket{\Em(t)} = \sum_N P(N, t) \ket{N}.
\end{align}
%
This is familiar to anyone who has done some many-particle quantum mechanics---a state is the linear combination of states with different numbers of particles.
Likewise, we can check directly that
%
\begin{align}
    a\ket{N} &= N \ket{N},&
    a^\dagger\ket{N} &= \ket{N+1},&
    [a^\dagger, a] \equiv a^\dagger a - a a^\dagger = 1.
\end{align}
%
The last term is the familiar canonical commutation relation for bosons.
One may consider fermionic systems, representing exclusion processes where only a finite number of particles can occupy a state, but we will keep to bosons here for the sake of simplicity.

If we have multi species systems, the generalization is straight forward.
The states, operators and their commutations are denote
%
\begin{align}
    \ket{N_1, \dots N_k}
    = \ket{\bm N}
    , &&
    a_i,\, a_j^\dagger, &&
    [a_i^\dagger, a_j] & = \delta_{ij}.
\end{align}
%
The equation for the state vector can now be written in a form similar to the Schrödinger equation (in imaginary time),
%
\begin{align}
    \odv[]{}{t} \ket{\Em(t)} = \Ell[a^\dagger, a] \ket{\Em(t)},
\end{align}
%
which has the formal solution
%
\begin{align}
    \ket{\Em(t)}
    = 
    e^{\Ell[a^\dagger, a](t-t_0)}
    \ket{\Em(t_0)}.
\end{align}
%


In addition to the \emph{ket} vectors, $\ket{\cdot}$, we introduce the corresponding \emph{dual vectors}, or \emph{bra}'s, denoted $\bra{\cdot}$.
Since $\ket{N} = z^N$, we can consider $\ket{\Em(t)} = \sum_N P(N, t)\ket{N}$ as an element of the vector space of analytic functions, $V$, with $P(N, t)$ as the components.
\note{I guess the vector space is actually $\ell^1$, since $\sum |P(N)| = 1$? So a different space than Fokk space, which is $\ell^2$. LC: I have seen $V$ being referred to as a (stochastic) Fock space in the literature, e.g. Doi's original 1976 paper.}
When we have such a vector space, \emph{Riesz representation theorem} says that the space of all linear operators from $V$ to real numbers, $V^*$, is also a vector space, with a basis of elements $\bra{M} \in V^*$ such that
%
\begin{align}
    \braket{M|N} = \delta_{M,N}.
\end{align}
%
This is the dual space, and $\bra{M}$ is the basis for dual vectors.
The inner product is defined as
%
\begin{align}
    \braket{\cdot| N} &= \int \dd z \, \cdot z^N, \\
    \braket{M| \cdot} & = \int \dd z \, \frac{ (-1)^M }{ M! } \delta^{(M)}(z) \cdot.
\end{align}
%
Here, $\delta^{(M)}(z)$ is the M'th derivative of the Dirac delta, defined by partial integration,
%
\begin{align}
    \int \dd z \, \delta^{(N)} f(z)
    = 
    (-1)^N f^{(N)}(0)
    \implies 
    \braket{M | \Em{(t)}} = P(N, t).
\end{align}
%

The (leftward) action of the operators on the dual vector can be deduced as follows:
%
\begin{align}
    \bra{M} \left(a \ket{N}\right)
    &= N \braket{M|N-1} = N \delta_{M, N-1}
    = (M - 1) \braket{M+1 | N}\\
    \implies 
    \bra{M}a & = (M + 1)\bra{M + 1}
\end{align}
%
and
\begin{align}
    \bra{M} \left(a^\dagger \ket{N}\right)
    &= \braket{M|N+1} = \delta_{M, N-1}
    = \braket{M-1 | N}\\
    \implies 
    \bra{M}a^\dagger & = \bra{M - 1}.
\end{align}
% 
From this, we get a simpler representation of the dual basis in terms of the creation operators and the dual of the `empty' state
%
\begin{align}
    \bra{M} &= \bra{0} \frac{ a^M }{ M! }, &
    \braket{M|\cdot} &= \bra{0} \frac{ a^M }{ M! }\ket{\cdot}, &
    \braket{0|\cdot} = \int \dd z \, \delta(z)\cdot.
\end{align}
%
We may then write moments as
%
\begin{align}
    \E{N^k(t)} 
    =
    \sum_N \left(z \odv{  }{ z }\right)^k z^N \big|_{z = 1} P(N, t)
    =
    \sum_N \left(z \odv{  }{ z }\right)^k\Em(z, t)\big|_{z = 1}
    =
    \int \dd z \, \delta(z - 1)
    \sum_N \left(z \odv{  }{ z }\right)^k\Em(z, t).
\end{align}
%
We may Taylor-expand this around $z = 0$, so
%
\begin{align}
    \E{N^k(t)} 
    =
    \sum_m
    \int \dd z \, \frac{ \delta(z) }{ m! }
    \odv[m]{  }{ z }
    \sum_N \left(z \odv{  }{ z }\right)^k\Em(z, t)
    =
    \bra{0}e^a (a^\dagger a)^k \ket{\Em(t)}.
\end{align}
%
We call
%
\begin{align}
    \bra{0}e^a = \bra{0} + \bra{1} + \bra{2} + ...
    \equiv
    \bra{\star}
\end{align}
%
the \emph{coherent state}, or \emph{abyss}. It is invariant under $a^\dagger$ from the right. In practice, it evaluates whatever appears to its right at $z=1$. Thus, it is always the case that $\bra{\star} e^{\mathcal{L}t}\ket{\Em(0)}=1$ by normalisation.

\note{
    Maybe a bit on the differences from QM, this is not quite clear to me.
    The operators are not unitary, right?
    The correspondence between the vectors and dual vectors is more complicated...
    Is the dagger actually hermitian conjugate?
    etc... LC: indeed, the operators are not unitary ($\mathcal{L}$ is stochastic rather than self-adjoint). Also, dagger here does not indicate hermitian conjugate, as can be seen by writing the operators explicitly in matrix form, after having represented $\Em$ as a vector of probabilities. This is just borrowing notation from QM but I agree that it is a bit suggestive. Maybe we can discuss this a bit in the next lecture, although I will admit I can only say so much about this before I get confused myself...
    }
\note{This might also be interesting in that context:
\url{https://arxiv.org/pdf/2302.10778}}
