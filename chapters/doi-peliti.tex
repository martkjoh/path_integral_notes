The response and Onsager-Machlup formalism is useful for perturbative calculations, however, it relies on some assumption which do not always hold.
Most importantly, if we cannot describe the small time-step as a Gaussian process, or/and the random variable in question cannot be modeled as continuous, then it fails.
Examples of discrete processes are birth and death processes with low population number, or discrete jump processes, automata or reaction-diffusion processes.
In this case, we can instead use the Doi-Peliti formalism.
To derive this, we start with the master equation, as we derived in \autoref{section: master equation}.

\section{Discrete master equation}

We will consider one or more random variables $N(t) \in \NN$.
The corresponding master equation is
%
\begin{align}
    \partial_t P(N, t)
    = 
    \sum_{N'}
    \left\{
        W_t(N | N') P(N', t)
        - W_t(N'|N)P(N, t)
    \right\}.
\end{align}
%
To recap, the first term is the gain term, while the second is the loss-term.
The $W$'s are transition rates, infinite dimensional matrices that contain the rates at which one state transition into another.
The result is an infinite set of ODE's for the probabilities $P(N, t)$.
Some examples of these processes are

\subsection*{Extinction
$
\left(
    \parbox{20mm}{
    \centering
    \begin{fmfgraph*}(8,2)
        \setval
        \fmfleft{i}
        \fmfright{o}
        \fmf{fermion,label=$\epsilon$}{i,o}
        \fmflabel{$A$}{i}
        \fmflabel{$\emptyset$}{o}
    \end{fmfgraph*}
    }
\right)
$
}
This process represents one set of particles, labeled $A$, which decays at a constant rate, $\epsilon$, to nothing $(\emptyset)$, as illustrated by the arrow above.
$N(t)$ then represent the total numbers of particles of type $A$,
and the transition rates are the decay rate, $\epsilon$, times the number of particles.
The corresponding master equation is represented the rate at which the system transitions into, minus the rate out of, state $N$,
%
\begin{align}
    \partial_t P(N, t) = 
    \epsilon
    \left[
        (N + 1) P(N+1, t)
        - 
        N P(N, t)
    \right].
\end{align}
%
The steady-state is a probability distribution $P(N, t) = P_S(N, t)$ so that $\partial_t P(N) = 0$, and the only such state is
%
\begin{align}
    P_S(N) = \delta_{N,0},
\end{align}
%
the ``emtpy state''.

We may derive a differential equation for the expected number of particles, $\E{N(t)}$, by ``applying'' $\sum_N N \cdot$ to  the master equation, which yields
%
\begin{align}
    \partial_t
    \underbrace{\sum_N N P(N, t)}_{\E{N(t)}}
    &= 
    \epsilon
    \sum_{N}
    \bigg\{
        \underbrace{N(N + 1)}_{(N + 1)^2-(N+1)}P(N+1,t)
        - NP(N,t) k
    \bigg\}\\
    & = 
    \epsilon
    \left\{
        \E{N(t)^2} - \E{N(t)} - \E{N(t)^2}
    \right\}
    =     \epsilon \E{N(t)},
\end{align}
%
This is solved by $\E{N(t)} = N(0) e^{-\epsilon t}$.


\subsection*{Spontaneous creation
$
\left(
    \parbox{20mm}{
    \centering
    \begin{fmfgraph*}(8,2)
        \setval
        \fmfleft{i}
        \fmfright{o}
        \fmf{fermion,label=$\gamma$}{i,o}
        \fmflabel{$\emptyset$}{i}
        \fmflabel{$A$}{o}
    \end{fmfgraph*}
    }
\right)
$
}

This is the converse process, where particles of species A pop into existence at a constant rate $\gamma$ independent of the current occupation, with the master equation
%
\begin{align}
    \partial_t P(N, t)
    = 
    \gamma 
    \left\{
        P(N - 1, t) - P(N, t)
    \right\}.
\end{align}
%
One may notice this has no steady state, as this would require $P_S(N) = P_S(N+1)$, a uniform distribution on the natural numbers, which is unnormalizable.


\subsection*{Dynamic steady state
$
\left(
    \parbox{34mm}{
    \centering
    \begin{fmfgraph*}(22,2)
        \setval
        \fmfleft{i}
        \fmfright{o}
        \fmf{fermion,label=$\gamma$}{i,c1}
        \fmf{phantom,tension=2}{c1,c2}
        \fmf{fermion,label=$\epsilon$}{c2,o}
        \fmflabel{$\emptyset$}{i}
        \fmflabel{$B$}{o}
        \fmfv{l=$A$,l.a=0,l.d=.03w}{c1}
    \end{fmfgraph*}
    }
\right)
$
}

A generic property of the processes considered here is that we may always consider the combination of two (concurrent) processes by adding together the right-hand sides of the corresponding master equations.
The master equation for the spontaneous creation and annihilation process is therefore
%
\begin{align}
    \partial_t P(N, t) = 
    \epsilon
    \left[
        (N + 1) P(N+1, t)
        - 
        N P(N, t)
    \right]
    +
    \gamma 
    \left\{
        P(N - 1, t) - P(N, t)
    \right\}.
\end{align}
%
Assuming $P(N<0) = 0$ gives the steady state,
%
\begin{align}
    P_S(N) = \frac{ P_S(0) }{ N! } \left(\frac{ \gamma }{ \epsilon }\right)^N
    =
    \frac{ 1 }{ N! } \left(\frac{ \gamma }{ \epsilon }\right)^N e^{-\gamma/\epsilon},
\end{align}
%
which is the Poission distribution. This statistical steady-state may be defined as ``dynamic'' since the occupation number for a given realisation of the process may still vary over time.\\

A couple of other processes, and their rates, are

\subsection*{Coagulation
$
\left(\,
    \parbox{22mm}{
    \centering
    \begin{fmfgraph*}(8,2)
        \setval
        \fmfleft{i}
        \fmfright{o}
        \fmf{fermion,label=$\lambda$}{i,o}
        \fmflabel{$2A$}{i}
        \fmflabel{$A$}{o}
    \end{fmfgraph*}
    }
\right)
$
}

%
\begin{align}
    W(N-1|N) = \lambda \binom{N}{2}
\end{align}
%


\subsection*{Transmutation
$
\left(
    \parbox{20mm}{
    \centering
    \begin{fmfgraph*}(8,2)
        \setval
        \fmfleft{i}
        \fmfright{o}
        \fmf{fermion,label=$\gamma$}{i,o}
        \fmflabel{$A$}{i}
        \fmflabel{$B$}{o}
    \end{fmfgraph*}
    }
\right)
$
}

%
\begin{align}
    W(N_A-1,N_B+1|N_A,N_B) 
    = \tau N_A
\end{align}
%

\subsection*{Spawning
$
\left(
    \parbox{30mm}{
    \centering
    \begin{fmfgraph*}(22,6)
        \setval
        \fmfleft{i,i2}
        \fmfright{o,o2}
        \fmf{fermion}{i2,c,o2}
        \fmflabel{$s$}{c}
        \fmffreeze
        \fmf{fermion}{c,o}
        \fmflabel{$A$}{i2}
        \fmflabel{$A$}{o}
        \fmflabel{$B$}{o2}
    \end{fmfgraph*}
    }
\right)
$
}

%
\begin{align}
    W(N_A,N_B+1|N_A,N_B) 
    = s N_A
\end{align}
%


\subsection*{Catalysis
$
\left(
    \parbox{34mm}{
    \centering
    \begin{fmfgraph*}(28,6)
        \setval
        \fmfleft{i,i2}
        \fmfright{o,o2}
        \fmf{fermion}{i2,c,o2}
        \fmflabel{$t$}{c}
        \fmffreeze
        \fmf{fermion,left=.2}{i,c,o}
        \fmflabel{$A$}{i2}
        \fmflabel{$B$}{o2}
        \fmflabel{$C$}{i}
        \fmflabel{$C$}{o}
    \end{fmfgraph*}
    }
\right)
$
}

%
\begin{align}
    W(N_A-1,N_B+1, N_C|N_A,N_B, N_C) 
    = t N_A N_C
\end{align}
%

\subsection*{General process}

A general process has $k$ species, labeled $A_i$, with $i\in\{1, \dots k\}$, $j_i$ label the number of species $i$ that is needed for the process, while $\ell_i$ are the number of $A_i$ that are output by the reaction (reactants and products), which has rate $r$.
The corresponding diagram is
%
\begin{align}
    \parbox{30mm}{
    \centering
    \begin{fmfgraph*}(22,15)
        \setval
        \fmfleft{i0,i1,i2,i3}
        \fmfright{o0,o1,o2,o3}
        \fmf{fermion}{i0,c,o0}
        \fmf{phantom}{i1,c,o1}
        \fmf{fermion}{i2,c,o2}
        \fmf{fermion}{i3,c,o3}
        \fmfv{d.f=empty,d.shape=circle,d.size=5mm,l=$r$,l.d=0cm}{c}
        \fmflabel{$j_kA_k$}{i0}
        \fmflabel{$\vdots$}{i1}
        \fmflabel{$j_2A_2$}{i2}
        \fmflabel{$j_1A_1$}{i3}
        \fmflabel{$\ell_kA_k$}{o0}
        \fmflabel{$\vdots$}{o1}
        \fmflabel{$\ell_2A_2$}{o2}
        \fmflabel{$\ell_1A_1$}{o3}
    \end{fmfgraph*}
    }
\end{align}
%
and the transition rates are
%
\begin{align}
    W(N_1-\ell_1 + j_1, \dots N_k - \ell_k + J_k| N_1, \dots N_k)
    = r \binom{N_1}{\ell_1} \cdots \binom{N_k}{\ell_k}
\end{align}
%


\subsection*{Processes in space}

We can immediately generalise to processes in space, by considering $A_i$ to be a particle on lattice site $i$ and $N_i(t)$ the instantaneous occupation of said site.
If we consider an equal probability of jumping left or right, hopping rate $D / h^2$, where $h$ is the lattice spacing, then the master equation is
%
\begin{align}
    \partial_t P(\bm N, t) 
    = 
    \frac{ D }{ h^2 } \sum_{j\in \Z}\sum_{\E{x,j}}
    \left[ (N_x + 1) P(\bm N + 1_x - 1_j,t) - N_x P(\bm N, t) \right].
\end{align}
%
Here, $\bm N$ is the vector of occupation number at the different sites, and $1_x$ is the vector with only $1$ at site $x$, $0$ elsewhere, and the bracket means sum over nearest neighbors.
Applying $\sum_{\bm N} N_x \cdot$, we get
%
\begin{align}
    \partial_t \E{N_x(t)} = D \nabla^2_x \E{N_x(t)},
\end{align}
%
the (discrete) diffusion equation. In this sense, the familiar diffusion equation captures only some of the information contained in the original master equation, namely that about average occupation.


\section{Probability generating function}

We define the function
%
\begin{align}
    \Em(z, t) = \sum_N z^N P(N, t)
\end{align}
%
for a single particle species, or for many species,
%
\begin{align}
    \Em(z_1, ... z_K, t) 
    =
    \Em(\bm z, t) 
    = \sum_{\bm N} z^{N_1}\cdots z^{N_k} P(\bm N, t).
\end{align}
%
This is called the probability generating function (PGF) because we can obtain the probability of having the state $N = \ell$ by taking derivatives of it evaluated at $z=0$,
%
\begin{align}
    P(\ell, t) = \frac{ 1 }{ \ell! } \partial_z^\ell \Em(z, t)\big|_{z = 0}.
\end{align}
%
Notice that, for all $t$, $M(z = 1, t) = 1$, which is an important property stemming from normalisation of $P(N,t)$.

If we again consider the master equation for the birth and death process,
%
\begin{align}
    \partial_t P(N, t) = 
    \epsilon
    \left[
        (N + 1) P(N+1, t)
        - 
        N P(N, t)
    \right]
    +
    \gamma 
    \left\{
        P(N - 1, t) - P(N, t)
    \right\},
\end{align}
%
and apply $\sum_N z^N \cdot$, we get a single PDE for $\Em$,
%
\begin{align}
    \partial_t \Em(z, t) 
    &= 
    \epsilon 
    \sum_N 
    \left\{
        \underbrace{z^N (N + 1)}_{\partial_z z^{N + 1}} P(N + 1, t)
        - 
        \underbrace{z^N N}_{z \partial_z z^N} P(N, t )
    \right\}
    +
    \gamma \sum_N
    z^N
    \left\{
         P(N-1, t) - P(N, t)
    \right\}\\
    & = \epsilon (1 - z) \partial_z \Em(z, t)
    + \gamma (z - 1) \Em(z, t).
\end{align}
%
If we consider steady state, $\partial_t \Em = 0$, we get 
%
\begin{align}
    \Em_S(z) = \Em_0 e^{\gamma /\epsilon z}
    \implies
    P_S(N) = 
    \frac{ 1 }{ N! }
    \left(\frac{ \gamma }{ \epsilon }\right)^N 
    e^{-\gamma/\epsilon},
\end{align}
%
as before.

\begin{framed}
    \textit{Exercise:}
    Show that, for the generic process described above, the equation of the generating function is
    %
    \begin{align}
        \partial_t \Em(\bm z, t)
        = \frac{ r }{ \prod_k \ell_k ! }
        \left[
            \left(
                \prod_k z^{j_k}_k \partial_{z_k}^{\ell_k}
            \right)
            -
            \left(
                \prod_k z^{\ell_k}_k \partial_{z_k}^{\ell_k}
            \right)
        \right]
        \Em(\bm z, t).
    \end{align}
    %
\end{framed}

We end this section with a note: although by definition the generating function is analytic in $z$, as it is a power series, the result differential equations may have non-analytic solutions, which must be discarded.
We may also obtain moments from the generating functions by exploiting its analytical properties to write
%
\begin{align}
    \E{N^k(t)}
    = 
    \sum_N \left(z \odv{  }{ z }\right)^k z^N P(N) \big|_{z=1}
    = 
    \left. \left(z \odv{  }{ z }\right)^k \Em(z, t) \right|_{z=1}
\end{align}
Note that the derivatives are evaluate at $z=1$, not $z=0$ as before.



\section{Second quantization}

With the definition and properties of the probability generating function clear, we now introduce a new formalism called \emph{second quantization}, due to its close relation to the similarly named formalism of quantum field theory.
This formalism is a mapping of the objects from the previous chapter to a new notation (first suggested by Masao Doi in 1976).
This mapping has the following dictionary,

\begin{table}[h]
    \centering
    \begin{tabular}{c|c}
        Usual & 2nd quant. \\[.1cm]
        \hline\\[-.2cm]
        Factors of $z^N$ & particle state $\ket{N}$ \\[.1cm]
        Multiplication with $z$ & Creation operator $a^\dagger$\\[.1cm]
        Derivative $\odv{  }{ z }$ & Annihilation operator $a$ \\[.1cm]
        PGF $\Em(z, t)$ & System state $\ket{\Em(t)}$
    \end{tabular}
\end{table}

If the PGF is $\Em(z, t) = z^M$, then $P(N, t) = \delta_{N,M}$ and $N(t)=M$ with probability one. For a generic statistical superposition of occupations, we may write
%
\begin{align}
    \ket{\Em(t)} = \sum_N P(N, t) \ket{N}.
\end{align}
%
This is familiar to anyone who has done some many-particle quantum mechanics---a state is the linear combination of states with different numbers of particles.
Likewise, we can check directly that
%
\begin{align}
    a\ket{N} &= N \ket{N},&
    a^\dagger\ket{N} &= \ket{N+1},&
    [a^\dagger, a] \equiv a^\dagger a - a a^\dagger = 1.
\end{align}
%
The last term is the familiar canonical commutation relation for bosons.
One may consider fermionic systems, representing exclusion processes where only a finite number of particles can occupy a state, but we will keep to bosons here for the sake of simplicity.

If we have multi species systems, the generalization is straight forward.
The states, operators and their commutations are denote
%
\begin{align}
    \ket{N_1, \dots N_k}
    = \ket{\bm N}
    , &&
    a_i,\, a_j^\dagger, &&
    [a_i^\dagger, a_j] & = \delta_{ij}.
\end{align}
%
The equation for the state vector can now be written in a form similar to the Schrödinger equation (in imaginary time),
%
\begin{align}
    \odv[]{}{t} \ket{\Em(t)} = \Ell[a^\dagger, a] \ket{\Em(t)},
\end{align}
%
which has the formal solution
%
\begin{align}
    \label{eq: shrodinger}
    \ket{\Em(t)}
    = 
    e^{\Ell[a^\dagger, a](t-t_0)}
    \ket{\Em(t_0)}.
\end{align}
%


In addition to the \emph{ket} vectors, $\ket{\cdot}$, we introduce the corresponding \emph{dual vectors}, or \emph{bra}'s, denoted $\bra{\cdot}$.
Since $\ket{N} = z^N$, we can consider $\ket{\Em(t)} = \sum_N P(N, t)\ket{N}$ as an element of the vector space of analytic functions, $V$, with $P(N, t)$ as the components.
When we have such a vector space, \emph{Riesz representation theorem} says that the space of all linear operators from $V$ to real numbers, $V^*$, is also a vector space, with a basis of elements $\bra{M} \in V^*$ such that
%
\begin{align}
    \braket{M|N} = \delta_{M N}.
\end{align}
%
This is the dual space, and $\bra{M}$ is the basis for dual vectors.
The inner product is defined as
%
\begin{align}
    \braket{\cdot| N} &= \int \dd z \, \cdot z^N, \\
    \braket{M| \cdot} & = \int \dd z \, \frac{ (-1)^M }{ M! } \delta^{(M)}(z) \cdot.
\end{align}
%
Here, $\delta^{(M)}(z)$ is the $M^\text{th}$ derivative of the Dirac delta, defined by partial integration,
%
\begin{align}
    \int \dd z \, \delta^{(N)} f(z)
    = 
    (-1)^N f^{(N)}(0)
    \implies 
    \braket{M | \Em{(t)}} = P(N, t).
\end{align}
%

The (leftward) action of the operators on the dual vector can be deduced as follows:
%
\begin{align}
    \bra{M} \left(a \ket{N}\right)
    &= N \braket{M|N-1} = N \delta_{M, N-1}
    = (M - 1) \braket{M+1 | N}\\
    \implies 
    \bra{M}a & = (M + 1)\bra{M + 1}
\end{align}
%
and
\begin{align}
    \bra{M} \left(a^\dagger \ket{N}\right)
    &= \braket{M|N+1} = \delta_{M, N-1}
    = \braket{M-1 | N}\\
    \implies 
    \bra{M}a^\dagger & = \bra{M - 1}.
\end{align}
% 
From this, we get a simpler representation of the dual basis in terms of the creation operators and the dual of the `empty' state
%
\begin{align}
    \bra{M} &= \bra{0} \frac{ a^M }{ M! }, &
    \braket{M|\cdot} &= \bra{0} \frac{ a^M }{ M! }\ket{\cdot}, &
    \braket{0|\cdot} = \int \dd z \, \delta(z)\cdot.
\end{align}
%
We may then write moments as
%
\begin{align}
    \E{N^k(t)} 
    =
    \sum_N \left(z \odv{  }{ z }\right)^k z^N \big|_{z = 1} P(N, t)
    =
    \sum_N \left(z \odv{  }{ z }\right)^k\Em(z, t)\big|_{z = 1}
    =
    \int \dd z \, \delta(z - 1)
    \sum_N \left(z \odv{  }{ z }\right)^k\Em(z, t).
\end{align}
%
We may Taylor-expand this around $z = 0$, so
%
\begin{align}
    \E{N^k(t)} 
    =
    \sum_m
    \int \dd z \, \frac{ \delta(z) }{ m! }
    \odv[m]{  }{ z }
    \sum_N \left(z \odv{  }{ z }\right)^k\Em(z, t)
    =
    \bra{0}e^a (a^\dagger a)^k \ket{\Em(t)}.
\end{align}
%
We call
%
\begin{align}
    \bra{0}e^a = \bra{0} + \bra{1} + \bra{2} + ...
    \equiv
    \bra{\star}
\end{align}
%
the \emph{coherent state}, or \emph{abyss}. It is invariant under $a^\dagger$ from the right. In practice, it evaluates whatever appears to its right at $z=1$. Thus, it is always the case that $\bra{\star} e^{\mathcal{L}t}\ket{\Em(0)}=1$ by normalisation.


\note{If you are interested in the link between stochastic processes and QM, this might be worth a look:\url{https://arxiv.org/pdf/2302.10778}}


\subsection*{Identity operators}

We now expand the time evolution of the state vector into small time-intervals, similar to our approach earlier.
The formal solution of the Schrodinger equations, \autoref{eq: shrodinger}, is then
%
\begin{align}
    \ket{\Em(t)} &= e^{\Ell[a^\dagger, a](t - t_0)} \ket{\Em(t_0)}
    \\
    & = \lim_{\Delta t \rightarrow 0} \left(\one + \Delta t \Ell\right)^n
    \ket{\Em(t_0)},
\end{align}
%
where $\Delta t = (t - t_0) / n$.
We then insert an indentiy operator, $\one$, between each ``time step''
%
\begin{align}
    \lim_{\Delta t \rightarrow 0} 
    \one 
    \left(\one + \Delta t \Ell\right) \one
    \left(\one + \Delta t \Ell\right) \one \dots
    \one
    \left(\one + \Delta t \Ell\right) \one
    \ket{\Em(t_0)},
\end{align}
%
One can verify, as $\ket{N}$ is a basis so that $\bra{M}\ket{N} = \delta_{NM}$, that
%
\begin{align}
    \sum_N \ket{N} \bra{M} = \one.
\end{align}
%
In quantum mechanics, relations as these are known as ``completeness relations''.
If we write this in terms of eigenvalues of the operators, this makes life easier.
If we have $\hat A \one \hat B$, and can write $\one = \sum_k \ket{e_k}\bra{e_k}$, where $\hat A \ket{e_k} = \mu_k^{(A)}\ket{e_k}$ and $\bra{e_k} \hat B = \mu_k^{(B)}\ket{e_k}$, then we can get rid of the operators, and instead just be left with complex numbers, which are easier to deal with.
We first need to normal order $\Ell$. This means to write all annihilation operators to the right, and creation to the left.
For example, using $[a^\dagger, a] = 1$, we can rearrange
%
\begin{align}
    a a^\dagger a = a^\dagger a^2 + a.
\end{align}
%
With this, if we find the right eigenvectors of $a$ and left eigenvectors of $a^\dagger$, we have what we need.
If we define $\ket{\phi} = e^{\phi a} \ket{0}$ and $\bra{\psi} = \bra{0}e^{\phi a^\dagger}$ where $\phi$ and $\psi$ are complex numbers, then one can verify that
%
\begin{align}
    a \ket{\phi} &=  \phi \ket{\phi}, &
    \bra{\psi} a^\dagger & = \psi \bra{\psi}.
\end{align}
%
Furthermore,
%
\begin{align}
    \ket{\phi} \bra{\phi} = e^{\phi a^\dagger} \ket{0}\bra{0} e^{\phi^* a}
    = \sum_{NM} \frac{ 1 }{ N! } \phi^N {\phi^*}^M \ket{N}\bra{M}.
\end{align}
%
Thus, using the Gaussian integral, 
%
\begin{align}
    \frac{1}{\pi}
    \int \dd \mathrm{Re} \phi \, \dd \mathrm{Im} \phi \,
    e^{- \phi \phi^*} \ket{\phi} \bra{\phi}
    &=
    \sum_{NM} \ket{N}\bra{M} \frac{ 1 }{ N! } 
    \int \dd \mathrm{Re} \phi\, \dd \mathrm{Im} \phi \,
    \phi^N {\phi^*}^M\\
    & = \sum_{NM} \ket{N}\bra{M} \frac{ 1 }{ N! } 
    \int \dd \rho \dd \theta e^{-\rho^2} \rho^{N+M+1} e^{i\theta(N-M)}
    =  \sum_{NM} \ket{N}\bra{M}  = \one.
\end{align}
%
When inserting this form of the ideas, we will get terms of the form
%
\begin{align}
    \bra{\phi_{j+1}} \left( \one + \Delta t \Ell[a^\dagger, a] \right) \ket{\phi_j}
    = \left(1 + \Delta t \Ell[\phi_{j+1}^*,\phi_j]\right) e^{\phi_j\phi_{j+1}^*}.
\end{align}
%
Here, $j$ indexes the time step where the identity was inserted, $t_j = t_0 + j \Delta t$.
With all this, we get, denoting $\int \dd \mathrm{Re} \phi \, \dd \mathrm{Im} / \pi = \dd \phi$,
%
\begin{align}
    e^{\Ell[a^\dagger, a](t - t_0)} 
    & = 
    \lim_{\Delta t \rightarrow 0} 
    \one_{N}
    \left(\one + \Delta t \Ell\right) \one_{N-1}
    \left(\one + \Delta t \Ell\right) \one_{N-2} \dots
    \one_1
    \left(\one + \Delta t \Ell\right) \one_0\\
    &= 
    \lim_{\Delta t \rightarrow 0} \int 
    \left[\prod_{K = 0}^{N}e^{\phi_K\phi_K^*} \dd \phi_K\right]
    \ket{\phi_N}\bra{\phi_0^*}
    \prod_{j=0}^{N-1} 
    \left[
        e^{-\phi_j \phi_{j+1}^*}\left(1 + \Delta t \Ell[\phi_{j+1}^*,\phi_j]\right) 
    \right].
\end{align}
%
In the continuum limit, we get a path integral,
%
\begin{align}
    e^{\Ell[a^\dagger, a](t - t_0)} 
    = 
    \int \D \phi \, 
    \ket{\phi(t)}\bra{\phi^*(t_0)}
    \exp \left\{ - \phi(t_0)\phi^*(t_0) 
    - \int \dd t \left[
        \phi^*(t) \odv{  }{ t } \phi(t) - \Ell[\phi^*(t), \phi(t)]
    \right]\right\}.
\end{align}
%

\subsection*{Measurement}

In general, we are interested in evaluate expression of the form
%
\begin{align}
    \bra{\star} \hat O e^{\Ell(t_2 - t_1)} \hat P e^{\Ell(t_1 - t_0)} \ket{\Em(t_0)}.
\end{align}
%
Here, the process is initialized at $t_0$, then evolved to $t_1$, where it is perturbed by $\hat P$, evolves to $t_2$, where an observable $\hat O$ is measured.
Starting with the initialization, this has the form
%
\begin{align}
    \braket{\phi^*(t_0)|\Em(t_0)}
    =
    \bra{\phi^*(t_0)} \hat I[a^\dagger] \ket{0}
    =
    I[\phi^*(t_0)],
\end{align}
%
where $I$ is the initialization operator, which states which particles are in the state at $t = t_0$.

\begin{framed}
    \textit{Exercise:}
    What is $\hat I$ for a state initiated with a Poisson distribution of mean $\lambda$? What is $    \braket{\phi^*(t_0)|\Em(t_0)}$?
\end{framed}


Next, the perturbation takes the form
%
\begin{align}
    \ket{\Em(t_2)} 
    =
    e^{\Ell(t_2 - t_1)} \hat P[a^\dagger, a] e^{\Ell(t_1 - t_0) }
    \ket{\Em(t_0)},
\end{align}
%
where $\hat P$ is assumed normal ordered.
This then appears as a term in the exponential,
%
\begin{align}
    \ket{\Em(t_2)} 
    =
    \int \D \phi \, 
    \ket{\phi(t_1)}\braket{\phi^*(t_0)| \Em(t_0)}
    \exp \left\{ 
        - \phi(t_0)\phi^*(t_0)
        - 
        P(\phi(t_1)^*, \phi(t_1))
        -
        \int_{t_0}^{t_2} \dd t \cdots
     \right\}.
\end{align}
%

Lastly, to measure moments, we found that we must chose $\Oh = (a^\dagger a)^K$.
Writing $ \ket{\Em(t_2)} = \int \D \phi e^{A} \ket{\phi(t_2)}  $ we have
%
\begin{align}
    \bra{\star} (a^\dagger a)^K \ket{\Em(t_2)}
    = 
    \int \D \phi e^{A} \bra{\star} (a^\dagger a)^K  \ket{\phi(t_2)}.
\end{align}
%
Then, to evaluate the operators, we must normal order them.
We have
%
\begin{align}
    \bra{\star} (a^\dagger a)^K  \ket{\phi(t_2)}
    = 
    \sum_{\ell=0}^K \left\{ K \atop \ell \right\} \phi(t_2)^{\ell} e^{\phi(t_2)}.
\end{align}
%
The brackets indicate Stirling numbers of the second kind. 
The appears from shuffling the creation and annihilation operators to active normal ordering~\footnote{See \url{https://en.wikipedia.org/wiki/Stirling_numbers_of_the_second_kind}}.
Fork $K = 1$, we get simply $\phi(t_2)e^{\phi(t_2)}$.

The initial conditions $t_0$ can be neglected if we let $t_0\rightarrow - \infty$.
So, letting a system evolve from $t_0$ to $t_1$, when it is initialized with $\hat I$, the to $t_2$, where it is perturbed, and then finally measure $\hat \Oh$ at $t_3$, we get
%
\begin{align}
    &
    \bra{\star} \hat O e^{\Ell(t_3 - t_2)}  \hat P e^{\Ell(t_2 - t_1)} \hat I e^{\Ell(t_1 - t_0)} \ket{\Em(t_0)}\\
    & \equiv
    \E{
    \left(
        \sum_{\ell = 0}^K \left\{ K \atop \ell \right\} \phi(t_3)^{\ell} e^{\phi(t_3)}
    \right)
    P[\phi^*(t_2),\phi(t_2)] I[\phi^*(t_1),\phi(t_1)]
    }
    \\\nonumber
    & = \int \D \phi \D \phi^* \,
    P[\phi^*(t_2),\phi(t_2)] I[\phi^*(t_1),\phi(t_1)]
    \left(
        \sum_{\ell = 0}^K \left\{ \ell \atop K \right\} \phi(t_3)^{\ell} e^{\phi(t_3)}
    \right)
    \exp \left\{ 
    -\int_{-\infty}^{t_3} \dd t \, 
    \left(\phi^*(t)\odv{  }{ t }\phi(t) - \Ell[\phi^*(t), \phi(t)]\right)
    \right\}.
\end{align}
%
Finally, we can remove the $\phi(t_3)$ factor by using a ``Doi-shift''.
We define $\phi^* = \tilde \phi + 1$, which means a term from the integral in the exponential absorb it.

\section{The shortcut}

This derivation has given us a recipe for constructing path integrals:
\begin{itemize}
    \item Write the master equation (ME)
    \item Rewrite it using the probability generating function (PGF) $\Em(z, t)$, which gives the PDE $\partial_t\Em = \Ell[z, \partial_z] \Em$.
    \item Use the 2nd quantization (2Q), substituting $z, \pdv{  }{ z }$ for $a^\dagger, a$, and normal order $\Ell[a^\dagger, a]$.
    \item Write the action (A) as $A = \int \dd t \left(\tilde \varphi \odv{  }{ t } \varphi - \Ell[\tilde \phi + 1, \phi]\right)$.
    \item Expectation values are now evaluated using the path integral (PI), $\E{\cdot } = \int \D \phi \, \cdot \,  e^{A} $.
\end{itemize}
%
Now, we can replace $\cdot$ by operators such as $\Oh$ or $I$.
This can be calculated, as we have seen earlier for other path integrals, by isolating the Gaussian part using perturbation theory.
Computations are then done with Wicks theorem and Feynman diagrams.

As and example, consider spontaneous decay
\begin{itemize}
    \item ME: $\odv{  }{ t }P_N(t) = r(N+1)P_{N+1}(t) - r NP_N(t)$.
    \item PGF: $\partial_t \Em(z, t) = r (1 - z)\partial_z \Em(z, t)$.
    \item 2Q: $\Ell[\hat a, a] = r (1 - a^\dagger)a $.
    \item A: $A = \int \dd t \left(\tilde \phi \odv{  }{ t }\phi - r\tilde \phi \phi\right) = \int \dd \omega \tilde \phi(-i\omega + r)\phi$.
\end{itemize}
Consider the expectation value of a single particle, so $\Oh = \phi(t_1)$, given that we inject one at $t_0$, $I = \phi^*(t_0) = \tilde \phi(t_0) + 1$.
%
\begin{align}
    \E{\phi(t_1) ( \tilde \phi(t_0) + 1) }
    =
    \E{\phi(t_1) \tilde \phi(t_0) } + \E{\phi(t_1)}.
\end{align}
%
Here, the second term is zero, while the first is the ``propagator'', 
%
\begin{align}
    \E{\phi(t_1)\tilde \phi(t_0)}
    =
    \parbox{32mm}{
    \centering
    \begin{fmfgraph*}(12,2)
        \setval
        \fmfleft{i}
        \fmfright{o}
        \fmf{fermion}{o,i}
        \fmflabel{$\tilde \phi(t_0)$}{o}
        \fmflabel{$\phi(t_1)$}{i}
    \end{fmfgraph*}
    }
    = 
    e^{- r(t_1 - t_0)}\theta(t_1 - t_0).
\end{align}
%

A few remarks are in order
\begin{itemize}
    \item The normalization condition $\Em(z = 1, t) = 1$ implies that $\Ell(\tilde \phi = 0, \phi) = 0$.
    \item The field $\phi$ is always evaluated at $t + \Delta t$, while $\phi$ is at $t$ in $\Ell$, and causality impies $\E{\phi(t) \tilde \phi(t + \Delta t)} = 0$.
    \item In Fourier space, $\E{\phi(\omega)\tilde \phi(\omega')} = \frac{ 2 \pi \delta(\omega + \omega) }{ -i \omega + r }$.
\end{itemize}


\section{Non-linearities}

As we have seen, we can add new processes to the master equation.
Consider branching, 
%
\begin{align}
    \odv{  }{ t } P_N(t) = \beta (N - 1)P_{N-1}(t ) - \beta N P_N(t).
\end{align}
%
This give $\hat \Ell_I = \beta (a - 1)a^\dagger a \implies \Ell_I = \beta \tilde \phi^2\phi + \beta \tilde \phi \phi$, so 
%
\begin{align}
    e^{A} 
    &= \exp \left\{ 
        \int\dd t
        \left[
            \tilde \phi \left(\odv{  }{ t } + r - \beta\right)\phi
            - \beta \tilde \phi^2 \phi
        \right]
     \right\}  \\
     &= \exp \left\{ 
        \int\dd t
            \tilde \phi \left(\odv{  }{ t } + r - \beta\right)\phi
     \right\}  
     \sum_n \frac{ \beta^n }{ n! }
     \left(- \int \dd t \tilde \phi^2 \phi\right)^n.
\end{align}
%
We see this process affects the decay rate.
We also get a non-linear term, whcih we note by a vertex,
%
\begin{align}
    - \beta \int \dd t \tilde \phi^2 \phi
    =
    \parbox{35mm}{
    \centering
    \begin{fmfgraph*}(20,6)
        \setval
        \fmfleft{i1,i2}
        \fmfright{o}
        \fmf{fermion, tension=2}{o,c}
        \fmf{fermion}{c,i1}
        \fmf{fermion}{c,i2}
        \fmflabel{$\phi(t)$}{o}
        \fmflabel{$\tilde \phi(t)$}{i1}
        \fmflabel{$\tilde \phi(t)$}{i2}
    \end{fmfgraph*}
    }
\end{align}
%
Perturbation calculations, for example of the second moments, given by
%
\begin{align}
    \E{\phi(t_1)^2 \tilde \phi(t_0) }_A
    &= 
    \int \dd t \E{\phi(t_1)^2 \tilde \phi(t)^2 \phi(t) \tilde \phi(t_0)}_0+\dots\\
    & =
    \parbox{25mm}{
    \centering
    \begin{fmfgraph*}(20,6)
        \setval
        \fmfleft{i1,i2}
        \fmfright{o}
        \fmf{fermion, tension=2}{o,c}
        \fmf{fermion}{c,i1}
        \fmf{fermion}{c,i2}
    \end{fmfgraph*}
    } + \dots
\end{align}
%
