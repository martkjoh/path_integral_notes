The response and Onsager-Machlup formalism is useful for perturbative calculations, however, it relies on some assumption which do not always hold.
Most importantly, if we cannot describe the small time-step as a Gaussian process, or/and the random variable in question cannot be modeled as continuous, then it fails.
Examples of discrete processes are birth and death processes with low population number, or discrete jump processes, automata or reaction-diffusion processes.
In this case, we can instead use the Doi-Peliti formalism.
To derive this, we start with the master equation, as we derived in \autoref{section: master equation}.

\section{Discrete master equation}

We will consider one or more random variables $N(t) \in \NN$.
The corresponding master equation is
%
\begin{align}
    \partial_t P(N, t)
    = 
    \sum_{N'}
    \left\{
        W_t(N | N') P(N', t)
        - W_t(N'|N)P(N, t)
    \right\}.
\end{align}
%
To recap, the first term is the gain term, while the second is the loss-term.
The $W$'s are transition rates, infinite dimensional matrices that contain the rates at which one state transition into another.
The result is an infinite set of ODE's for the probabilities $P(N, t)$.
Some examples of these processes are

\subsection*{Extinction
$
\left(
    \parbox{20mm}{
    \centering
    \begin{fmfgraph*}(8,2)
        \setval
        \fmfleft{i}
        \fmfright{o}
        \fmf{fermion,label=$\epsilon$}{i,o}
        \fmflabel{$A$}{i}
        \fmflabel{$\emptyset$}{o}
    \end{fmfgraph*}
    }
\right)
$
}
This process represents one set of particles, labeled $A$, which decays at a constant rate, $\epsilon$, to nothing $(\emptyset)$, as illustrated by the arrow above.
$N(t)$ then represent the total numbers of particles of type $A$,
and the transition rates are the decay rate, $\epsilon$, times the number of particles.
The corresponding master equation is represented the rate at which the system transitions into, minus the rate out of, state $N$,
%
\begin{align}
    \partial_t P(N, t) = 
    \epsilon
    \left[
        (N + 1) P(N+1, t)
        - 
        N P(N, t)
    \right].
\end{align}
%
The steady-state is a probability distribution $P(N, t) = P_S(N, t)$ so that $\partial_t P(N) = 0$, and the only such state is
%
\begin{align}
    P_S(N) = \delta_{N,0},
\end{align}
%
the ``emtpy state''.

We may derive a differential equation for the expected number of particles, $\E{N(t)}$, by ``applying'' $\sum_N N \cdot$ to  the master equation, which yields
%
\begin{align}
    \partial_t
    \underbrace{\sum_N N P(N, t)}_{\E{N(t)}}
    &= 
    \epsilon
    \sum_{N}
    \bigg\{
        \underbrace{N(N + 1)}_{(N + 1)^2-(N+1)}P(N+1,t)
        - NP(N,t) k
    \bigg\}\\
    & = 
    \epsilon
    \left\{
        \E{N(t)^2} - \E{N(t)} - \E{N(t)^2}
    \right\}
    =     \epsilon \E{N(t)},
\end{align}
%
This is solved by $\E{N(t)} = N(0) e^{-\epsilon t}$.


\subsection*{Spontaneous creation
$
\left(
    \parbox{20mm}{
    \centering
    \begin{fmfgraph*}(8,2)
        \setval
        \fmfleft{i}
        \fmfright{o}
        \fmf{fermion,label=$\gamma$}{i,o}
        \fmflabel{$\emptyset$}{i}
        \fmflabel{$A$}{o}
    \end{fmfgraph*}
    }
\right)
$
}

This is the converse process, where particles of species A pop into existence at a constant rate $\gamma$ independent of the current occupation, with the master equation
%
\begin{align}
    \partial_t P(N, t)
    = 
    \gamma 
    \left\{
        P(N - 1, t) - P(N, t)
    \right\}.
\end{align}
%
One may notice this has no steady state, as this would require $P_S(N) = P_S(N+1)$, a uniform distribution on the natural numbers, which is unnormalizable.


\subsection*{Dynamic steady state
$
\left(
    \parbox{34mm}{
    \centering
    \begin{fmfgraph*}(22,2)
        \setval
        \fmfleft{i}
        \fmfright{o}
        \fmf{fermion,label=$\gamma$}{i,c1}
        \fmf{phantom,tension=2}{c1,c2}
        \fmf{fermion,label=$\epsilon$}{c2,o}
        \fmflabel{$\emptyset$}{i}
        \fmflabel{$B$}{o}
        \fmfv{l=$A$,l.a=0,l.d=.03w}{c1}
    \end{fmfgraph*}
    }
\right)
$
}

A generic property of the processes considered here is that we may always consider the combination of two (concurrent) processes by adding together the right-hand sides of the corresponding master equations.
The master equation for the spontaneous creation and annihilation process is therefore
%
\begin{align}
    \partial_t P(N, t) = 
    \epsilon
    \left[
        (N + 1) P(N+1, t)
        - 
        N P(N, t)
    \right]
    +
    \gamma 
    \left\{
        P(N - 1, t) - P(N, t)
    \right\}.
\end{align}
%
Assuming $P(N<0) = 0$ gives the steady state,
%
\begin{align}
    P_S(N) = \frac{ P_S(0) }{ N! } \left(\frac{ \gamma }{ \epsilon }\right)^N
    =
    \frac{ 1 }{ N! } \left(\frac{ \gamma }{ \epsilon }\right)^N e^{-\gamma/\epsilon},
\end{align}
%
which is the Poission distribution. This statistical steady-state may be defined as ``dynamic'' since the occupation number for a given realisation of the process may still vary over time.\\

A couple of other processes, and their rates, are

\subsection*{Coagulation
$
\left(\,
    \parbox{22mm}{
    \centering
    \begin{fmfgraph*}(8,2)
        \setval
        \fmfleft{i}
        \fmfright{o}
        \fmf{fermion,label=$\lambda$}{i,o}
        \fmflabel{$2A$}{i}
        \fmflabel{$A$}{o}
    \end{fmfgraph*}
    }
\right)
$
}

%
\begin{align}
    W(N-1|N) = \lambda \binom{N}{2}
\end{align}
%


\subsection*{Transmutation
$
\left(
    \parbox{20mm}{
    \centering
    \begin{fmfgraph*}(8,2)
        \setval
        \fmfleft{i}
        \fmfright{o}
        \fmf{fermion,label=$\gamma$}{i,o}
        \fmflabel{$A$}{i}
        \fmflabel{$B$}{o}
    \end{fmfgraph*}
    }
\right)
$
}

%
\begin{align}
    W(N_A-1,N_B+1|N_A,N_B) 
    = \tau N_A
\end{align}
%

\subsection*{Spawning
$
\left(
    \parbox{30mm}{
    \centering
    \begin{fmfgraph*}(22,6)
        \setval
        \fmfleft{i,i2}
        \fmfright{o,o2}
        \fmf{fermion}{i2,c,o2}
        \fmflabel{$s$}{c}
        \fmffreeze
        \fmf{fermion}{c,o}
        \fmflabel{$A$}{i2}
        \fmflabel{$A$}{o}
        \fmflabel{$B$}{o2}
    \end{fmfgraph*}
    }
\right)
$
}

%
\begin{align}
    W(N_A,N_B+1|N_A,N_B) 
    = s N_A
\end{align}
%


\subsection*{Catalysis
$
\left(
    \parbox{34mm}{
    \centering
    \begin{fmfgraph*}(28,6)
        \setval
        \fmfleft{i,i2}
        \fmfright{o,o2}
        \fmf{fermion}{i2,c,o2}
        \fmflabel{$t$}{c}
        \fmffreeze
        \fmf{fermion,left=.2}{i,c,o}
        \fmflabel{$A$}{i2}
        \fmflabel{$B$}{o2}
        \fmflabel{$C$}{i}
        \fmflabel{$C$}{o}
    \end{fmfgraph*}
    }
\right)
$
}

%
\begin{align}
    W(N_A-1,N_B+1, N_C|N_A,N_B, N_C) 
    = t N_A N_C
\end{align}
%

\subsection*{General process}

A general process has $k$ species, labeled $A_i$, with $i\in\{1, \dots k\}$, $j_i$ label the number of species $i$ that is needed for the process, while $\ell_i$ are the number of $A_i$ that are output by the reaction (reactants and products), which has rate $r$.
The corresponding diagram is
%
\begin{align}
    \parbox{30mm}{
    \centering
    \begin{fmfgraph*}(22,15)
        \setval
        \fmfleft{i0,i1,i2,i3}
        \fmfright{o0,o1,o2,o3}
        \fmf{fermion}{i0,c,o0}
        \fmf{phantom}{i1,c,o1}
        \fmf{fermion}{i2,c,o2}
        \fmf{fermion}{i3,c,o3}
        \fmfv{d.f=empty,d.shape=circle,d.size=5mm,l=$r$,l.d=0cm}{c}
        \fmflabel{$j_kA_k$}{i0}
        \fmflabel{$\vdots$}{i1}
        \fmflabel{$j_2A_2$}{i2}
        \fmflabel{$j_1A_1$}{i3}
        \fmflabel{$\ell_kA_k$}{o0}
        \fmflabel{$\vdots$}{o1}
        \fmflabel{$\ell_2A_2$}{o2}
        \fmflabel{$\ell_1A_1$}{o3}
    \end{fmfgraph*}
    }
\end{align}
%
and the transition rates are
%
\begin{align}
    W(N_1-\ell_1 + j_1, \dots N_k - \ell_k + J_k| N_1, \dots N_k)
    = r \binom{N_1}{\ell_1} \cdots \binom{N_k}{\ell_k}
\end{align}
%


\subsection*{Processes in space}

We can immediately generalise to processes in space, by considering $A_i$ to be a particle on lattice site $i$ and $N_i(t)$ the instantaneous occupation of said site.
If we consider an equal probability of jumping left or right, hopping rate $D / h^2$, where $h$ is the lattice spacing, then the master equation is
%
\begin{align}
    \partial_t P(\bm N, t) 
    = 
    \frac{ D }{ h^2 } \sum_{j\in \Z}\sum_{\E{x,j}}
    \left[ (N_x + 1) P(\bm N + 1_x - 1_j,t) - N_x P(\bm N, t) \right].
\end{align}
%
Here, $\bm N$ is the vector of occupation number at the different sites, and $1_x$ is the vector with only $1$ at site $x$, $0$ elsewhere, and the bracket means sum over nearest neighbors.
Applying $\sum_{\bm N} N_x \bullet$, we get
%
\begin{align}
    \partial_t \E{N_x(t)} = D \nabla^2_x \E{N_x(t)},
\end{align}
%
the (discrete) diffusion equation. In this sense, the familiar diffusion equation captures only some of the information contained in the original master equation, namely that about average occupation.


\section{Probability generating function}

We define the function
%
\begin{align}
    \Em(z, t) = \sum_N z^N P(N, t)
\end{align}
%
for a single particle species, or for many species,
%
\begin{align}
    \Em(z_1, ... z_K, t) 
    =
    \Em(\bm z, t) 
    = \sum_{\bm N} z^{N_1}\cdots z^{N_k} P(\bm N, t).
\end{align}
%
This is called the probability generating function (PGF) because we can obtain the probability of having the state $N = \ell$ by taking derivatives of it evaluated at $z=0$,
%
\begin{align}
    P(\ell, t) = \frac{ 1 }{ \ell! } \partial_z^\ell \Em(z, t)\big|_{z = 0}.
\end{align}
%
Notice that, for all $t$, $M(z = 1, t) = 1$, which is an important property stemming from normalisation of $P(N,t)$.

If we again consider the master equation for the birth and death process,
%
\begin{align}
    \partial_t P(N, t) = 
    \epsilon
    \left[
        (N + 1) P(N+1, t)
        - 
        N P(N, t)
    \right]
    +
    \gamma 
    \left\{
        P(N - 1, t) - P(N, t)
    \right\},
\end{align}
%
and apply $\sum_N z^N \cdot$, we get a single PDE for $\Em$,
%
\begin{align}
    \partial_t \Em(z, t) 
    &= 
    \epsilon 
    \sum_N 
    \left\{
        \underbrace{z^N (N + 1)}_{\partial_z z^{N + 1}} P(N + 1, t)
        - 
        \underbrace{z^N N}_{z \partial_z z^N} P(N, t )
    \right\}
    +
    \gamma \sum_N
    z^N
    \left\{
         P(N-1, t) - P(N, t)
    \right\}\\
    & = \epsilon (1 - z) \partial_z \Em(z, t)
    + \gamma (z - 1) \Em(z, t).
\end{align}
%
If we consider steady state, $\partial_t \Em = 0$, we get 
%
\begin{align}
    \Em_S(z) = \Em_0 e^{\gamma /\epsilon z}
    \implies
    P_S(N) = 
    \frac{ 1 }{ N! }
    \left(\frac{ \gamma }{ \epsilon }\right)^N 
    e^{-\gamma/\epsilon},
\end{align}
%
as before.

\begin{framed}
    \textit{Exercise:}
    Show that, for the generic process described above, the equation of the generating function is
    %
    \begin{align}
        \partial_t \Em(\bm z, t)
        = \frac{ r }{ \prod_k \ell_k ! }
        \left[
            \left(
                \prod_k z^{j_k}_k \partial_{z_k}^{\ell_k}
            \right)
            -
            \left(
                \prod_k z^{\ell_k}_k \partial_{z_k}^{\ell_k}
            \right)
        \right]
        \Em(\bm z, t).
    \end{align}
    %
\end{framed}

We end this section with a note: although by definition the generating function is analytic in $z$, as it is a power series, the result differential equations may have non-analytic solutions, which must be discarded.
We may also obtain moments from the generating functions by exploiting its analytical properties to write
%
\begin{align}
    \E{N^k(t)}
    = 
    \sum_N \left(z \odv{  }{ z }\right)^k z^N P(N) \big|_{z=1}
    = 
    \left. \left(z \odv{  }{ z }\right)^k \Em(z, t) \right|_{z=1}
\end{align}
Note that the derivatives are evaluate at $z=1$, not $z=0$ as before.



\section{Second quantization}

With the definition and properties of the probability generating function clear, we now introduce a new formalism called \emph{second quantization}, due to its close relation to the similarly named formalism of quantum field theory.
This formalism is a mapping of the objects from the previous chapter to a new notation (first suggested by Masao Doi in 1976).
This mapping has the following dictionary,

\begin{table}[h]
    \centering
    \begin{tabular}{c|c}
        Usual & 2nd quant. \\[.1cm]
        \hline\\[-.2cm]
        Factors of $z^N$ & particle state $\ket{N}$ \\[.1cm]
        Multiplication with $z$ & Creation operator $a^\dagger$\\[.1cm]
        Derivative $\odv{  }{ z }$ & Annihilation operator $a$ \\[.1cm]
        PGF $\Em(z, t)$ & System state $\ket{\Em(t)}$
    \end{tabular}
\end{table}

If the PGF is $\Em(z, t) = z^M$, then $P(N, t) = \delta_{N,M}$ and $N(t)=M$ with probability one. For a generic statistical superposition of occupations, we may write
%
\begin{align}
    \ket{\Em(t)} = \sum_N P(N, t) \ket{N}.
\end{align}
%
This is familiar to anyone who has done some many-particle quantum mechanics---a state is the linear combination of states with different numbers of particles.
Likewise, we can check directly that
%
\begin{align}
    a\ket{N} &= N \ket{N},&
    a^\dagger\ket{N} &= \ket{N+1},&
    [a^\dagger, a] \equiv a^\dagger a - a a^\dagger = 1.
\end{align}
%
The last term is the familiar canonical commutation relation for bosons.
One may consider fermionic systems, representing exclusion processes where only a finite number of particles can occupy a state, but we will keep to bosons here for the sake of simplicity.

If we have multi species systems, the generalization is straight forward.
The states, operators and their commutations are denote
%
\begin{align}
    \ket{N_1, \dots N_k}
    = \ket{\bm N}
    , &&
    a_i,\, a_j^\dagger, &&
    [a_i^\dagger, a_j] & = \delta_{ij}.
\end{align}
%
The equation for the state vector can now be written in a form similar to the Schrödinger equation (in imaginary time),
%
\begin{align}
    \odv[]{}{t} \ket{\Em(t)} = \Ell[a^\dagger, a] \ket{\Em(t)},
\end{align}
%
which has the formal solution
%
\begin{align}
    \label{eq: shrodinger}
    \ket{\Em(t)}
    = 
    e^{\Ell[a^\dagger, a](t-t_0)}
    \ket{\Em(t_0)}.
\end{align}
%


In addition to the \emph{ket} vectors, $\ket{\bullet}$, we introduce the corresponding \emph{dual vectors}, or \emph{bra}'s, denoted $\bra{\bullet}$.
Since $\ket{N} = z^N$, we can consider $\ket{\Em(t)} = \sum_N P(N, t)\ket{N}$ as an element of the vector space of analytic functions, $V$, with $P(N, t)$ as the components.
When we have such a vector space, \emph{Riesz representation theorem} says that the space of all linear operators from $V$ to real numbers, $V^*$, is also a vector space, with a basis of elements $\bra{M} \in V^*$ such that
%
\begin{align}
    \braket{M|N} = \delta_{M N}.
\end{align}
%
This is the dual space, and $\bra{M}$ is the basis for dual vectors.
The inner product is defined as
%
\begin{align}
    \braket{\bullet| N} &= \int \dd z \, \bullet z^N, \\
    \braket{M| \bullet} & = \int \dd z \, \frac{ (-1)^M }{ M! } \delta^{(M)}(z) \bullet.
\end{align}
%
Here, $\delta^{(M)}(z)$ is the $M^\text{th}$ derivative of the Dirac delta, defined by partial integration,
%
\begin{align}
    \int \dd z \, \delta^{(N)} f(z)
    = 
    (-1)^N f^{(N)}(0)
    \implies 
    \braket{M | \Em{(t)}} = P(N, t).
\end{align}
%

The (leftward) action of the operators on the dual vector can be deduced as follows:
%
\begin{align}
    \bra{M} \left(a \ket{N}\right)
    &= N \braket{M|N-1} = N \delta_{M, N-1}
    = (M - 1) \braket{M+1 | N}\\
    \implies 
    \bra{M}a & = (M + 1)\bra{M + 1}
\end{align}
%
and
\begin{align}
    \bra{M} \left(a^\dagger \ket{N}\right)
    &= \braket{M|N+1} = \delta_{M, N-1}
    = \braket{M-1 | N}\\
    \implies 
    \bra{M}a^\dagger & = \bra{M - 1}.
\end{align}
% 
From this, we get a simpler representation of the dual basis in terms of the creation operators and the dual of the `empty' state
%
\begin{align}
    \bra{M} &= \bra{0} \frac{ a^M }{ M! }, &
    \braket{M|\bullet} &= \bra{0} \frac{ a^M }{ M! }\ket{\bullet}, &
    \braket{0|\bullet} = \int \dd z \, \delta(z)\bullet.
\end{align}
%
We may then write moments as
%
\begin{align}
    \E{N^k(t)} 
    =
    \sum_N \left(z \odv{  }{ z }\right)^k z^N \big|_{z = 1} P(N, t)
    =
    \sum_N \left(z \odv{  }{ z }\right)^k\Em(z, t)\big|_{z = 1}
    =
    \int \dd z \, \delta(z - 1)
    \sum_N \left(z \odv{  }{ z }\right)^k\Em(z, t).
\end{align}
%
We may Taylor-expand this around $z = 0$, so
%
\begin{align}
    \E{N^k(t)} 
    =
    \sum_m
    \int \dd z \, \frac{ \delta(z) }{ m! }
    \odv[m]{  }{ z }
    \sum_N \left(z \odv{  }{ z }\right)^k\Em(z, t)
    =
    \bra{0}e^a (a^\dagger a)^k \ket{\Em(t)}.
\end{align}
%
We call
%
\begin{align}
    \bra{0}e^a = \bra{0} + \bra{1} + \bra{2} + ...
    \equiv
    \bra{\star}
\end{align}
%
the \emph{coherent state}, or \emph{abyss}. It is invariant under $a^\dagger$ from the right. In practice, it evaluates whatever appears to its right at $z=1$. Thus, it is always the case that $\bra{\star} e^{\mathcal{L}t}\ket{\Em(0)}=1$ by normalization.
Thus, we write moments in the $2^\text{nd}$ quantized formalism as
%
\begin{align}
    \label{eq: moments in 2nd q}
    \E{N^k(t)} = \braket{\star | \left(a^\dagger a\right)^k | \Em(t)}
\end{align}
%


\note{If you are interested in the link between stochastic processes and QM, this might be worth a look:\url{https://arxiv.org/pdf/2302.10778}}


\subsection*{Identity operators}

We now expand the time evolution of the state vector into small time-intervals, similar to our approach earlier.
The formal solution of the Schrodinger equations, \autoref{eq: shrodinger}, is then
%
\begin{align}
    \ket{\Em(t)} &= e^{\Ell[a^\dagger, a](t - t_0)} \ket{\Em(t_0)}
    \\
    & = \lim_{\Delta t \rightarrow 0} \left(\one + \Delta t \Ell\right)^n
    \ket{\Em(t_0)},
\end{align}
%
where $\Delta t = (t - t_0) / n$.
We then insert an indentiy operator, $\one$, between each ``time step''
%
\begin{align}
    \lim_{\Delta t \rightarrow 0} 
    \one 
    \left(\one + \Delta t \Ell\right) \one
    \left(\one + \Delta t \Ell\right) \one \dots
    \one
    \left(\one + \Delta t \Ell\right) \one
    \ket{\Em(t_0)},
\end{align}
%
One can verify, as $\ket{N}$ is a basis so that $\bra{M}\ket{N} = \delta_{NM}$, that
%
\begin{align}
    \sum_N \ket{N} \bra{N} = \one.
\end{align}
%
In quantum mechanics, relations as these are known as ``completeness relations''.
If we write this in terms of eigenvalues of the operators, this makes life easier.
If we have $\hat A \one \hat B$, and can write $\one = \sum_k \ket{e_k}\bra{e_k}$, where $\hat A \ket{e_k} = \mu_k^{(A)}\ket{e_k}$ and $\bra{e_k} \hat B = \mu_k^{(B)}\ket{e_k}$, then we can get rid of the operators, and instead just be left with complex numbers, which are easier to deal with.
We first need to normal order $\Ell$. This means to write all annihilation operators to the right, and creation to the left.
For example, using $[a^\dagger, a] = 1$, we can rearrange
%
\begin{align}
    a a^\dagger a = a^\dagger a^2 + a.
\end{align}
%
With this, if we find the right eigenvectors of $a$ and left eigenvectors of $a^\dagger$, we have what we need.
If we define $\ket{\phi} = e^{\phi a} \ket{0}$ and $\bra{\psi} = \bra{0}e^{\phi a^\dagger}$ where $\phi$ and $\psi$ are complex numbers, then one can verify that
%
\begin{align}
    a \ket{\phi} &=  \phi \ket{\phi}, &
    \bra{\psi} a^\dagger & = \psi \bra{\psi}.
\end{align}
%
Furthermore,
%
\begin{align}
    \ket{\phi} \bra{\phi} = e^{\phi a^\dagger} \ket{0}\bra{0} e^{\phi^* a}
    = \sum_{NM} \frac{ 1 }{ N! } \phi^N {\phi^*}^M \ket{N}\bra{M}.
\end{align}
%
Thus, using the Gaussian integral, 
%
\begin{align}
    \frac{1}{\pi}
    \int \dd \mathrm{Re} \phi \, \dd \mathrm{Im} \phi \,
    e^{- \phi \phi^*} \ket{\phi} \bra{\phi}
    &=
    \sum_{NM} \ket{N}\bra{M} \frac{ 1 }{ N! } 
    \int \dd \mathrm{Re} \phi\, \dd \mathrm{Im} \phi \,
    \phi^N {\phi^*}^M\\
    & = \sum_{NM} \ket{N}\bra{M} \frac{ 1 }{ N! } 
    \int \dd \rho \dd \theta e^{-\rho^2} \rho^{N+M+1} e^{i\theta(N-M)}
    =  \sum_{N} \ket{N}\bra{N}  = \one.
\end{align}
%
When inserting this form of the identities, we will get terms of the form
%
\begin{align}
    \bra{\phi_{j+1}} \left( \one + \Delta t \Ell[a^\dagger, a] \right) \ket{\phi_j}
    = \left(1 + \Delta t \Ell[\phi_{j+1}^*,\phi_j]\right) e^{\phi_j\phi_{j+1}^*}.
\end{align}
%
Here, $j$ indexes the time step where the identity was inserted, $t_j = t_0 + j \Delta t$.
With all this, we get, denoting $\int \dd \mathrm{Re} \phi \, \dd \mathrm{Im} / \pi = \dd \phi$,
%
\begin{align}
    e^{\Ell[a^\dagger, a](t - t_0)} 
    & = 
    \lim_{\Delta t \rightarrow 0} 
    \one_{N}
    \left(\one + \Delta t \Ell\right) \one_{N-1}
    \left(\one + \Delta t \Ell\right) \one_{N-2} \dots
    \one_1
    \left(\one + \Delta t \Ell\right) \one_0\\
    &= 
    \lim_{\Delta t \rightarrow 0} \int 
    \left[\prod_{K = 0}^{N}e^{\phi_K\phi_K^*} \dd \phi_K\right]
    \ket{\phi_N}\bra{\phi_0^*}
    \prod_{j=0}^{N-1} 
    \left[
        e^{-\phi_j \phi_{j+1}^*}\left(1 + \Delta t \Ell[\phi_{j+1}^*,\phi_j]\right) 
    \right].
\end{align}
%
In the continuum limit, we get a path integral,
%
\begin{align}
    e^{\Ell[a^\dagger, a](t - t_0)} 
    = 
    \int \D \phi \, 
    \ket{\phi(t)}\bra{\phi^*(t_0)}
    \exp \left\{ - \phi(t_0)\phi^*(t_0) 
    - \int \dd t \left[
        \phi^*(t) \odv{  }{ t } \phi(t) - \Ell[\phi^*(t), \phi(t)]
    \right]\right\}.
\end{align}
%

\subsection*{Measurement}

In general, we are interested in evaluate expression of the form
%
\begin{align}
    \bra{\star} \hat O e^{\Ell(t_2 - t_1)} \hat P e^{\Ell(t_1 - t_0)} \ket{\Em(t_0)}.
\end{align}
%
Here, the process is initialized at $t_0$, then evolved to $t_1$, where it is perturbed by $\hat P$, evolves to $t_2$, where an observable $\hat O$ is measured.
Starting with the initialization, this has the form
%
\begin{align}
    \braket{\phi^*(t_0)|\Em(t_0)}
    =
    \bra{\phi^*(t_0)} \hat I[a^\dagger] \ket{0}
    =
    I[\phi^*(t_0)],
\end{align}
%
where $I$ is the initialization operator, which states which particles are in the state at $t = t_0$.

\begin{framed}
    \textit{Exercise:}
    What is $\hat I$ for a state initiated with a Poisson distribution of mean $\lambda$? What is $    \braket{\phi^*(t_0)|\Em(t_0)}$?
\end{framed}


Next, the perturbation takes the form
%
\begin{align}
    \ket{\Em(t_2)} 
    =
    e^{\Ell(t_2 - t_1)} \hat P[a^\dagger, a] e^{\Ell(t_1 - t_0) }
    \ket{\Em(t_0)},
\end{align}
%
where $\hat P$ is assumed normal ordered.
This then appears as a term in the exponential,
%
\begin{align}
    \ket{\Em(t_2)} 
    =
    \int \D \phi \, 
    \ket{\phi(t_1)}\braket{\phi^*(t_0)| \Em(t_0)}
    P[\phi(t_1)^*, \phi(t_1)]
    \exp \left\{ 
        - \phi(t_0)\phi^*(t_0)
        -
        \int_{t_0}^{t_2} \dd t \cdots
     \right\}.
\end{align}
%

Lastly, to measure moments, we found that we must chose $\Oh = (a^\dagger a)^K$.
Writing $ \ket{\Em(t_2)} = \int \D \phi e^{A} \ket{\phi(t_2)}  $ we have
%
\begin{align}
    \bra{\star} (a^\dagger a)^K \ket{\Em(t_2)}
    = 
    \int \D \phi e^{A} \bra{\star} (a^\dagger a)^K  \ket{\phi(t_2)}.
\end{align}
%
Then, to evaluate the operators, we must normal order them.
We have
%
\begin{align}
    \label{eq: nomral ordering moments}
    \bra{\star} (a^\dagger a)^K  \ket{\phi(t_2)}
    = 
    \sum_{\ell=0}^K \left\{ K \atop \ell \right\} \phi(t_2)^{\ell} e^{\phi(t_2)}.
\end{align}
%
The brackets indicate Stirling numbers of the second kind. 
The appears from shuffling the creation and annihilation operators to active normal ordering~\footnote{See \url{https://en.wikipedia.org/wiki/Stirling_numbers_of_the_second_kind}}.
Fork $K = 1$, we get simply $\phi(t_2)e^{\phi(t_2)}$.

The initial conditions $t_0$ can be neglected if we let $t_0\rightarrow - \infty$.
So, letting a system evolve from $t_0$ to $t_1$, when it is initialized with $\hat I$, the to $t_2$, where it is perturbed, and then finally measure $\hat \Oh$ at $t_3$, we get
%
\begin{align}
    &
    \bra{\star} \hat O e^{\Ell(t_3 - t_2)}  \hat P e^{\Ell(t_2 - t_1)} \hat I e^{\Ell(t_1 - t_0)} \ket{\Em(t_0)}\\
    & \equiv
    \E{
    \left(
        \sum_{\ell = 0}^K \left\{ K \atop \ell \right\} \phi(t_3)^{\ell} e^{\phi(t_3)}
    \right)
    P[\phi^*(t_2),\phi(t_2)] I[\phi^*(t_1),\phi(t_1)]
    }
    \\\nonumber
    & = \int \D \phi \D \phi^* \,
    P[\phi^*(t_2),\phi(t_2)] I[\phi^*(t_1),\phi(t_1)]
    \left(
        \sum_{\ell = 0}^K \left\{ \ell \atop K \right\} \phi(t_3)^{\ell} e^{\phi(t_3)}
    \right)
    \exp \left\{ 
    -\int_{-\infty}^{t_3} \dd t \, 
    \left(\phi^*(t)\odv{  }{ t }\phi(t) - \Ell[\phi^*(t), \phi(t)]\right)
    \right\}.
\end{align}
%
Finally, we can remove the $\phi(t_3)$ factor by using a ``Doi-shift''.
We define $\phi^* = \tilde \phi + 1$, which means a term from the integral in the exponential absorb it.

\section{The shortcut}

This derivation has given us a recipe for constructing path integrals:
\begin{itemize}
    \item Write the master equation (ME)
    \item Rewrite it using the probability generating function (PGF) $\Em(z, t)$, which gives the PDE $\partial_t\Em = \Ell[z, \partial_z] \Em$.
    \item Use the 2nd quantization (2Q), substituting $z, \pdv{  }{ z }$ for $a^\dagger, a$, and normal order $\Ell[a^\dagger, a]$.
    \item Write the action (A) as $A = \int \dd t \left(\tilde \varphi \odv{  }{ t } \varphi - \Ell[\tilde \phi + 1, \phi]\right)$.
    \item Expectation values are now evaluated using the path integral (PI), $\E{\bullet} = \int \D \phi \bullet  e^{A} $.
\end{itemize}
%
Now, we can replace $\cdot$ by operators such as $\Oh$ or $I$.
This can be calculated, as we have seen earlier for other path integrals, by isolating the Gaussian part using perturbation theory.
Computations are then done with Wicks theorem and Feynman diagrams.

As and example, consider spontaneous decay
\begin{itemize}
    \item ME: $\odv{  }{ t }P_N(t) = r(N+1)P_{N+1}(t) - r NP_N(t)$.
    \item PGF: $\partial_t \Em(z, t) = r (1 - z)\partial_z \Em(z, t)$.
    \item 2Q: $\Ell[\hat a, a] = r (1 - a^\dagger)a $.
    \item A: $A = \int \dd t \left(\tilde \phi \odv{  }{ t }\phi - r\tilde \phi \phi\right) = \int \dd \omega \tilde \phi(-i\omega + r)\phi$.
\end{itemize}
Consider the expectation value of a single particle, so $\Oh = \phi(t_1)$, given that we inject one at $t_0$, $I = \phi^*(t_0) = \tilde \phi(t_0) + 1$.
%
\begin{align}
    \E{\phi(t_1) ( \tilde \phi(t_0) + 1) }
    =
    \E{\phi(t_1) \tilde \phi(t_0) } + \E{\phi(t_1)}.
\end{align}
%
Here, the second term is zero, while the first is the ``propagator'', 
%
\begin{align}
    \E{\phi(t_1)\tilde \phi(t_0)}
    =
    \parbox{32mm}{
    \centering
    \begin{fmfgraph*}(12,2)
        \setval
        \fmfleft{i}
        \fmfright{o}
        \fmf{fermion}{o,i}
        \fmflabel{$\tilde \phi(t_0)$}{o}
        \fmflabel{$\phi(t_1)$}{i}
    \end{fmfgraph*}
    }
    = 
    e^{- r(t_1 - t_0)}\theta(t_1 - t_0).
\end{align}
%

A few remarks are in order
\begin{itemize}
    \item The normalization condition $\Em(z = 1, t) = 1$ implies that $\Ell(\tilde \phi = 0, \phi) = 0$.
    \item The field $\phi$ is always evaluated at $t + \Delta t$, while $\phi$ is at $t$ in $\Ell$, and causality impies $\E{\phi(t) \tilde \phi(t + \Delta t)} = 0$.
    \item In Fourier space, $\E{\phi(\omega)\tilde \phi(\omega')} = \frac{ 2 \pi \delta(\omega + \omega) }{ -i \omega + r }$.
\end{itemize}


\section{Non-linearities}

As we have seen, we can add new processes to the master equation.
Consider branching, 
%
\begin{align}
    \odv{  }{ t } P_N(t) = \beta (N - 1)P_{N-1}(t ) - \beta N P_N(t).
\end{align}
%
This give $\hat \Ell_I = \beta (a - 1)a^\dagger a \implies \Ell_I = \beta \tilde \phi^2\phi + \beta \tilde \phi \phi$, so 
%
\begin{align}
    e^{A} 
    &= \exp \left\{ 
        \int\dd t
        \left[
            \tilde \phi \left(\odv{  }{ t } + r - \beta\right)\phi
            - \beta \tilde \phi^2 \phi
        \right]
     \right\}  \\
     &= \exp \left\{ 
        \int\dd t
            \tilde \phi \left(\odv{  }{ t } + r - \beta\right)\phi
     \right\}  
     \sum_n \frac{ \beta^n }{ n! }
     \left(- \int \dd t \tilde \phi^2 \phi\right)^n.
\end{align}
%
We see this process affects the decay rate.
We also get a non-linear term, whcih we note by a vertex,
%
\begin{align}
    - \beta \int \dd t \tilde \phi^2 \phi
    =
    \parbox{35mm}{
    \centering
    \begin{fmfgraph*}(20,6)
        \setval
        \fmfleft{i1,i2}
        \fmfright{o}
        \fmf{fermion, tension=2}{o,c}
        \fmf{fermion}{c,i1}
        \fmf{fermion}{c,i2}
        \fmflabel{$\phi(t)$}{o}
        \fmflabel{$\tilde \phi(t)$}{i1}
        \fmflabel{$\tilde \phi(t)$}{i2}
    \end{fmfgraph*}
    }
\end{align}
%
Perturbation calculations, for example of the second moments, given by
%
\begin{align}
    \E{\phi(t_1)^2 \tilde \phi(t_0) }_A
    &= 
    \int \dd t \E{\phi(t_1)^2 \tilde \phi(t)^2 \phi(t) \tilde \phi(t_0)}_0+\dots\\
    & =
    \parbox{25mm}{
    \centering
    \begin{fmfgraph*}(20,6)
        \setval
        \fmfleft{i1,i2}
        \fmfright{o}
        \fmf{fermion, tension=2}{o,c}
        \fmf{fermion}{c,i1}
        \fmf{fermion}{c,i2}
    \end{fmfgraph*}
    } + \dots
\end{align}
%


\section{The coupon collector problem}

The coupon collector problem is stated as 
%
\begin{framed}\noindent
    \textit{Coupon Collector Problem} $N$ distinct object are sampled uniformly at random, with replacement, from an urn.
    How many attempts, on average, are required to sample every object at least once?
\end{framed}
%
We assume that the sampling happens as a Poisson process, with rate $s$.
To map this process to something more resembling a chemical process, think of it as a process where $N_C$ coupons, $\circ$ interact with $N_A$ ``agents'', $\bullet$, and undergo the reaction
%
\begin{align}
    \circ + \bullet \overset{s}{\longrightarrow} \circ.
\end{align}
%
This represents sampling an unsampled coupon.
The results of this process will be the same as the coupon collector, with a rate $s/N_A$.

\begin{framed}
    \textit{Exercise:}
    Use classical probabilistic arguments to show that the completion time, $C_N$ is given by the distribution
    %
    \begin{align}
        P(C_n \leq t) = \left[1 - e^{-St/N}\right]^N \underset{N\rightarrow \infty}{\sim} \exp \left\{ - \frac{ N }{ S } e^{-St/N} \right\}, 
        \quad \text{or} \quad
        P\left(\frac{ S }{ N } \left[C_n - \frac{ N }{ S } \ln \frac{ N }{ S }\right] \leq t\right) = e^{-e^{-t}}.
    \end{align}
    %
    Tip : use statistical independence $P(\mathrm{max}\{H_1, H_2\} \leq t) = P(H_1\leq t) \dots P(H_N\leq t)$, where $H_K$ is the time of sampling the $K^\text{th}$ coupon. (arrival time)
\end{framed}


\subsection*{Field theory}

The master equation corresponding to the process described above is
%
\begin{align}
    \partial_t P(N_C, N_A) = sN_A (N_C + 1) P(N_C + 1, N_A, t) - sN_A N_C P(N_C, N_A, t),
\end{align}
%
which gives the corresponding PGF equation
%
\begin{align}
    \Em(z_C, z_A, t) = s \left[ \partial_C z_A \partial_A - z_C \partial_C z_A \partial_A \right] \Em(z_C, z_A, t),
\end{align}
%
whre $\partial_{C/A} = \pdv{  }{ Z_{C/A} }$.
The corresponding $2^\text{nd}$ quantized version is
%
\begin{align}
    \partial_t \ket{\Em(t)} = s \big(\underbrace{a^\dagger_A a_A a_C - a^\dagger_A a^\dagger_C a_A a_C }_{\hat \Ell}\big) \ket{\Em(t)}.
\end{align}
%
Here, we have normal order the operators, using that $[a^\dagger_A, a_C] = [a^\dagger_C, a_A]  =  0$, while $[a^\dagger_A, a_A] = [a^\dagger_C, a_C] = 1$.
With this, using adding the regularization terms (or masses) $r_A$ and $r_C$, which we will set to zero subsequently, we get the action
%
\begin{align}
    A = - \int \dd t \left(
        \tilde\phi_A \left[\odv{  }{ t } + r_A\right] \phi_A + \tilde\phi_C \left[\odv{  }{ t }  + r_C\right]\phi_C + s \tilde \phi_A \tilde \phi_C \phi_A \phi_C + s \tilde \phi_C \phi_C\phi_A
    \right).
\end{align}
%
We split this into the bare and interaction action, $A = A_0 + A_I$.
Then, 
%
\begin{align}
    \E{\bullet}_0 &= \int \D \phi_A \D\phi_C \bullet e^{A_0} , &
    \E{\bullet} = \E{\bullet \, e^{A_I}}_0 = \sum_n \E{\bullet\, \frac{ A_I^n }{ n! } }_0.
\end{align}
%
The bare propagators are
%
\begin{align}
    \E{\phi_A(t) \tilde \phi_A(t')} 
    &= \theta(t - t') e^{-r_A(t - t')}
    =
    \parbox{18mm}{
        \centering
        \begin{fmfgraph*}(10,4)
            \setval
            \fmfleft{i}
            \fmfright{o}
            \fmf{wiggly}{i,o}
            \fmflabel{$t$}{i}
            \fmflabel{$t'$}{o}
        \end{fmfgraph*}
    },\\
    \E{\phi_C(t) \tilde \phi_C(t')} 
    &= \theta(t - t') e^{-r_C(t - t')}
    =
    \parbox{18mm}{
        \centering
        \begin{fmfgraph*}(10,4)
            \setval
            \fmfleft{i}
            \fmfright{o}
            \fmf{plain}{i,o}
            \fmflabel{$t$}{i}
            \fmflabel{$t'$}{o}
        \end{fmfgraph*}
    },\\
    \E{\phi_A(t) \tilde \phi_C(t')} 
    & = \E{\phi_C(t) \tilde \phi_A(t')}  = 0.
\end{align}
%
The interaction vertices are
%
\begin{align}
    A_I: \quad
    \parbox{14mm}{
        \centering
        \begin{fmfgraph*}(14,4)
            \setval
            \fmfleft{i1,i2}
            \fmfright{o1,o2}
            \fmf{plain}{i2,c,o2}
            \fmffreeze
            \fmf{wiggly}{c,o1}
        \end{fmfgraph*}
    },
    \quad
    \parbox{14mm}{
        \centering
        \begin{fmfgraph*}(14,4)
            \setval
            \fmfleft{i1,i2}
            \fmfright{o1,o2}
            \fmf{plain}{i2,c,o2}
            \fmffreeze
            \fmf{wiggly}{c,o1}
            \fmf{wiggly}{c,i1}
        \end{fmfgraph*}
    }.
\end{align}
%
With this, we can calculate the effective decay rate for $c$:
%
\begin{align}
    \E{\phi_C(t)\phi^\dagger_C(0)\phi_A^\dagger(0)}
    =
    \E{\phi_C(t)\tilde\phi_C(0)\tilde\phi_A(0)}
    +
    \underbrace{\E{\phi_C(t)\tilde\phi_C(0)}}_{e^{-r_Ct}}
    +
    \underbrace{\E{\phi_C(t)\tilde\phi_A(0)}}_{ = 0}
    +
    \underbrace{\E{\phi_C(t)}}_{ = 0}.
\end{align}
%
We are left with
%
\begin{align}
    \E{\phi_C(t)\tilde\phi_C(0)\tilde \phi_A(0)}
    =
    \parbox{14mm}{
        \centering
        \begin{fmfgraph*}(14,4)
            \setval
            \fmfleft{i1,i2}
            \fmfright{o1,o2}
            \fmf{plain}{i2,c,o2}
            \fmffreeze
            \fmf{wiggly}{c,o1}
            \fmfblob{.25w}{c}
        \end{fmfgraph*}
    }
    =
    \parbox{14mm}{
        \centering
        \begin{fmfgraph*}(14,4)
            \setval
            \fmfleft{i1,i2}
            \fmfright{o1,o2}
            \fmf{plain}{i2,c,o2}
            \fmffreeze
            \fmf{wiggly}{c,o1}
        \end{fmfgraph*}
    }
    +
    \parbox{18mm}{
        \centering
        \begin{fmfgraph*}(18,4)
            \setval
            \fmfleft{i1,i2}
            \fmfright{o1,o2}
            \fmf{plain}{i2,c1,c2,o2}
            \fmffreeze
            \fmf{wiggly,right}{c1,c2}
            \fmf{wiggly}{c2,o1}
        \end{fmfgraph*}
    }
    +
    \parbox{22mm}{
        \centering
        \begin{fmfgraph*}(22,4)
            \setval
            \fmfleft{i1,i2}
            \fmfright{o1,o2}
            \fmf{plain}{i2,c1,c2,c3,o2}
            \fmffreeze
            \fmf{wiggly,right}{c1,c2,c3}
            \fmf{wiggly}{c3,o1}
        \end{fmfgraph*}
    }
    + \dots.
\end{align}
%
Here, we have used the fact that we cannot have closed loops from the four-point vertex, due to causality.
This can be rearranged to give
%
\begin{align}
    \parbox{14mm}{
        \centering
        \begin{fmfgraph*}(14,4)
            \setval
            \fmfleft{i1,i2}
            \fmfright{o1,o2}
            \fmf{plain}{i2,c,o2}
            \fmffreeze
            \fmf{wiggly}{c,o1}
            \fmfblob{.25w}{c}
        \end{fmfgraph*}
    }
    =
    e^{-r_Ct} \sum_n (-s)^n \int_0^t \dd t_1  e^{-r_A t_1} \int_0^{t_1} \dd t_2  e^{-r_A (t_2 - t_1)} \dots \int_0^{t_{n-1}} \dd t_n  e^{-r_A (t_{n} - t_{n-1})} 
    \underset{r_A=r_C = 0}{\longrightarrow} e ^{-st} - 1,
\end{align}
%
Thus, altogether,
%
\begin{align}
    \lim_{r_A,r_C\rightarrow 0}
    \E{\phi_C(t)\phi^\dagger_C(0)\phi_A^\dagger(0)}
    = e^{-st}.
\end{align}
%
\begin{framed}
    \textit{Exercise:} Consider what happens if we have $K$ agents looking for coupons. Convince yourself that
    %
    \begin{align}
        \lim_{r_A,r_C\rightarrow 0}
        \E{\phi_C(t)\phi^\dagger_C(0)\left(\phi_A^\dagger(0)\right)^K}
        = e^{-Kst}.
    \end{align}
    %
\end{framed}

We could compute more moments, but that's not what we are after.
We want the probability that there are exactly zero particles of type $C$ at some time $t$ after initialization.
Consider the Abyss $\bra{\star} = \bra{\star}_A\otimes\bra{\star}_C$.
Since $\bra{\star}_C = \bra{0}_C e^{a_c}$, if we binom $\hat \Oh = e^{-a_c}\frac{ a^m }{ M! }$, then
%
\begin{align}
    \E{\Oh(t)} = \bra{\star} \hat \Oh \ket{\Em(t)} 
    = \bra{0}e^{a_C} e^{-a_C} \frac{ a_C^M }{ M! }\ket{\Em(t)}
    = \braket{M|\Em(t)} = P({M_c})(t),
\end{align}
%
where $P({M_c})(t)$ is the probability that we have $M_C$ of particle $M$ at time $C$.
The object $\Oh$ thus projects down on the desired number state.

%\todo[inline]{Why do we do this? And should it be $e^{\rho a^\dagger}$?}
We assume the initialization is a Poisson distribution with mean $\rho$,
%
\begin{align}
    \ket{\Em(0)} = 
    \sum_{M=0}^\infty \frac{e^{-\rho} \rho^M}{M!} (\phi_c^\dagger(0))^M | 0 \rangle 
    &= e^{-\rho} e^{\rho \phi_c^\dagger(0)} |0\rangle = 
    e^{\rho \tilde \phi_c(0)}\ket{0}.
\end{align}
%
This choice is motivated by the fact that the resulting expression for the probability density of the completion time is directly found to be Gumbel. We might also initialise with a specific number $N$ of initial coupons, however in that case we would need to take a limit $N \to \infty$ and rescale/recenter the distribution to obtain Gumbel (see Chapter 6 of \href{https://spiral.imperial.ac.uk/handle/10044/1/104386}{this thesis}).
From now on we leave the spawning of an A particle implicit, keeping in mind that this amounts to all C particles acquiring an effective decay rate $s$.
Then, 
%
\begin{align}
    P_M(t) = \frac{ 1 }{ M! } \E{e^{-\phi_C(t)}\phi_C(t)^M e^{\rho \tilde \phi_C(0)}}
    &=
    \sum \frac{ 1 }{ M! } \sum_{n,m} \frac{ (-1)^n \rho^m }{ n! m! } 
    \underbrace{ \E{\phi_C(t)^{n+M}\tilde \phi_C(0)^m}}_{m! \E{\phi_C(t)\tilde \phi_C(0)}\delta_{m,M+n}}\\
    & = \frac{ 1 }{ M! } \rho^M e^{-Mst - \rho e^{-st}}.
\end{align}
%
We find the expectation value by considering
%
\begin{align}
    \E{\phi_C(t)^{n+M}\tilde \phi_C(0)^m} = 
    n + M
    \parbox{15mm}{
        \centering
        \begin{fmfgraph*}(14,10)
            \setval
            \fmfleft{i1,i2,i3}
            \fmfright{o1,o2,o3}
            \fmf{plain}{i1,c,o1}
            \fmf{phantom}{i2,c,o2}
            \fmf{plain}{i3,c,o3}
            \fmfv{l=$\vdots$,l.a=20,l.d=.1w}{i2}
            \fmfv{l=$\vdots$,l.a=160,l.d=.1w}{o2}
            \fmfblob{.4w}{c}
        \end{fmfgraph*}
    } m
\end{align}
%
By using that the propagator is $e^{-st}$, and that we can only connect uninterrupted straight lines, and there must be equal many straight lines on each side, as they cannot terminate, we get the result above when considering the combinatorics.
%
Note that, even if we had kept the presence of an $A$ particle explicit, diagrams involving the component
\parbox{15mm}{
    \centering
    \begin{fmfgraph*}(15,3)
        \setval
        \fmfleft{i1,i2}
        \fmfright{o1,o2}
        \fmf{plain}{i1,c1,c2,o1}
        \fmf{plain}{i2,d1,d2,o2}
        \fmffreeze
        \fmf{wiggly}{c1,d2}
    \end{fmfgraph*}
}
with no incoming or outgoing $A$-species propagators would not contribute since $\langle \phi_A(t)\phi_A(t')\rangle$ has no connected component.

Finally, the chance that we complete the coupons exactly at time $t$ is equal to the probability flux into the empty state, which in turn equals the time derivative of the state probability (since there is no outflux)
%
\begin{align}
    R_0(t) = \odv{  }{ t } P_{M=0}(t) = s \rho e^{-st - \rho e^{-st}}.
\end{align}
%
This is the ``Gumbel distribution'', which you will often find in the form obtained after rescaling time such that $s=1$ and setting $\rho=1$.


\section{Doi-Peliti process in space}

Consider biased hopping in on the lattice $i\in \Z$, with rates $\ell$ to the left and $r$ to the right.
This follows the master equation %\todo{the sum over $i$ is right?}
%
\begin{align}
    &\odv{  }{ t } P(\{n_i\}, t) 
    = \\\nonumber &
    \sum_i \left[
        -(\ell + r) n_i P(\{n_i\}, t)
        + r (n_i + 1) P(\{..., n_i +1, n_{i+1}-1\}, t)
        + \ell (n_i + 1) P(\{..., n_{i-1} + 1, n_{i}+1, ... \}, t)
    \right].
\end{align}
%
This is similar to our earlier master equations, only now, the species index indicate spatial location.
The corresponding $2^\text{nd}$ quantized equation is
%
\begin{align}
    \partial_t \ket{\Em(t)}
    &=
    \sum_{i\in\Z} \left[
        -(\ell + r)a_i^\dagger a_i + r a_i a_{i+1}^\dagger + \ell r a_i a_{i-1}^\dagger 
    \right]\\
    & = 
    \sum_{i\in\Z} \left[
        \frac{ \ell +  }{ 2 } (a_{i + 1}^\dagger + a_{i - 1}^\dagger - 2 a_{i}^\dagger)
         + \frac{ r - \ell }{ 2 } (a_{i + 1}^\dagger - a_{i - 1}^\dagger)
        \right],
\end{align}
%
and the Doi-Peliti action is
%
\begin{align}
    A = 
    - \int \dd t \sum_{i}
    \left[
        \tilde \phi_i(t) \odv{  }{ t } \phi_i(t)
        - \frac{ \ell + r }{ 2 } \left(\tilde \phi_{i + 1}(t) + \tilde \phi_{i - 1}(t)
        - 2 \tilde \phi_{i}(t)
        \right)\phi_i(t)
        - (r - \ell ) \frac{ \tilde \phi_{i + 1}(t) - \tilde \phi_{i - 1}(t)}{ 2 }
        \phi_i(t)
    \right].
\end{align}
%
We see the term showing up here are finite-difference derivatives.
If we introduce the lattice spacing $h$, then take the limit $h\rightarrow 0$ while $hr, h\ell \rightarrow \const$, we get
%
\begin{align}
    A 
    & = - \int \dd t \dd x\left[
        \tilde \phi(t, x) \odv{  }{ t } \phi(t, x)
        - \frac{ 1 }{ 2 } h(\ell + r) \left(\partial_x^2 \tilde \phi(t, x)\right)\phi(t, x)
        - (r - \ell) \left(\partial_x \tilde \phi(t, x)\right)\phi(t, x)
    \right]\\
    & = 
    - \int \dd t \dd x\left[
        \tilde \phi(t, x) 
        \left(
            \odv{  }{ t } \phi(t, x)
            + D \partial_x^2 \phi(t, x)
            - v \partial_x \phi(t, x)
        \right)
    \right].
\end{align}
%
In the last line, we integrated by parts and defined the mean velocity and diffusion constants, $v$ and $D$.
This is a quadratic action, and we can therefor find expectation values exactly.
For example, the mean occupation at position $x$ after initializing a particle at time $t_0$ and position $x_0$ is
%
\begin{align}
    \E{\phi(t, x)\tilde\phi(t_0, x_0)}
    = \frac{ 1}{ \sqrt{ 2 \pi D (t - t_0)   } }
    \exp \left\{ - \frac{ (x - x_0 - v[t - t_0])^2 }{ 2 D (t - t_0) } \right\}.
\end{align}
%
Or, in Fourier space,
%a
\begin{align}
    \E{\phi(\omega, k)\tilde \phi(\omega', k')}
    = \frac{ 2\pi \delta(\omega + \omega')2\pi\delta(k+k') }{ -i\omega + Dk^2 -iv\omega }.
\end{align}
%$
A quick comment: upon performing the Doi shift we have effectively gotten rid of the daggered (creator) fields, so the relevant propagator for our field theory is now $\E{\phi(t, x)\tilde\phi(t_0, x_0)}$. Strictly speaking, this is not quite the same as $\E{\phi(t, x)\phi^\dagger(t_0, x_0)}$, which answers the question ``how many particles do I expect to see at $(x,t)$ given that I put down a single particle at $(x_0,t_0)$''. In fact the two differ by the factor $\E{\phi(t, x)}$ corresponding to the mean particle number in the absence of explicit initialisation, which vanishes whenever particles cannot simply spontaneously come into existence (e.g.\ in this example). To avoid confusion, one should always construct observables of interest using the full $\phi^\dagger$, then perform the Doi-shift and finally evaluate the ensuing expectations (typically a sum of products of fields) based on particular form of the ``shifted'' propagator $\E{\phi(t, x)\tilde\phi(t_0, x_0)}$.

Next, if we consider the correlation between two points, if we put down $n_0$ particle at $t_0, x_0$, we have
%
\begin{align}
    &C(x_0x_1,x_2,t_0,t_1,t_2)\\
    &=
    \left\langle
        \phi^\dagger(t_2, x_2)\phi(t_2, x_2)
        \phi^\dagger(t_1, x_1)\phi(t_1, x_1)
        \left[
            \phi^\dagger(t_0, x_0)
        \right]^{n_0}
    \right\rangle\\
    & =
    \binom{n_0}{1}
    \parbox{35mm}{
        \centering
        \begin{fmfgraph*}(15,8)
            \setval
            \fmfleft{i}
            \fmfright{o}
            \fmf{plain}{i,c,o}
            \fmflabel{$x_2\comma t_2$}{i}
            \fmfv{l=$x_1\comma t_1$,l.a=90}{c}
            \fmflabel{$x_0\comma t_0$}{o}
            \fmfdot{c}
        \end{fmfgraph*}
    }
    +
    \binom{n_0}{1}
    \parbox{35mm}{
        \centering
        \begin{fmfgraph*}(15,8)
            \setval
            \fmfleft{i}
            \fmfright{o}
            \fmf{plain}{i,c,o}
            \fmflabel{$x_2\comma t_2$}{i}
            \fmfv{l=$x_1\comma t_1$,l.a=90}{c}
            \fmflabel{$x_0\comma t_0$}{o}
            \fmfdot{c}
        \end{fmfgraph*}
    }
    +
    2
    \binom{n_0}{2}
    \parbox{30mm}{
        \centering
        \begin{fmfgraph*}(10,4)
            \setval
            \fmfleft{i}
            \fmfright{o}
            \fmf{plain}{i,o}
            \fmflabel{$x_1\comma t_1$}{i}
            \fmflabel{$x_0\comma t_0$}{o}
        \end{fmfgraph*}\\
        \begin{fmfgraph*}(10,8)
            \setval
            \fmfleft{i}
            \fmfright{o}
            \fmf{plain}{i,o}
            \fmflabel{$x_2\comma t_2$}{i}
            \fmflabel{$x_0\comma t_0$}{o}
        \end{fmfgraph*}
    }
\end{align}
%
In the last line, we usd that $(\phi^\dagger)^{n_0} = \sum_k \binom{n_0}{k}\tilde \phi^k$.
Either of the two first diagrams vanish, depending on $t_1>t_2$ or $t_1<t_2$.
This gives, assuming the latter is true,
%
\begin{align}
    C(x_0x_1,x_2,t_0,t_1,t_2)
    =
    &\binom{n_0}{1} \E{\phi(t_2,x_2)\tilde \phi(t_1,x_1)}\E{\phi(t_1,x_1)\tilde \phi(t_0,x_0)}\\
    &+ 2\binom{n_0}{2} \E{\phi(t_2,x_2)\tilde \phi(t_0,x_0)}\E{\phi(t_1,x_1)\tilde \phi(t_0,x_0)}.
\end{align}
%
The two different terms represent two different physical scenarios--- sequential visits by one particle, or independent visits by different particles. Note in particular that, when $n_0=1$ meaning that a single particle is initialised into the system, only the first term survives. This term also vanishes if we evaluate the expression at $t_2=t_1$ but $x_2 \neq x_1$, consistent with the intuition that point particles can 
only be in one place at any given time (and are not ``spread out'').

\subsection*{Run and tumble}

We now consider RnT dynamics by introducing a second species $\psi$ with $v' = - v$.
A particle ``tumbles'' by shifting from the right to the left moving, or opposite at rate $\tau$.
We assume they also decays at the same rate.
This gives the following action
%
\begin{align}
    A = 
    - \int \dd t \dd x\bigg[
        &
        \tilde \phi(t, x) 
        \left(
            \odv{  }{ t }
            + D \partial_x^2
            - v \partial_x 
        \right)\phi(t, x)
        +
        \tilde \psi(t, x) 
        \left(
            \odv{  }{ t }
            + D \partial_x^2
            - v \partial_x 
        \right)\psi(t, x)\\
        &+ \tau
        \left\{
            \tilde\phi(t, x) \psi(t, x)
            + \tilde \psi(t, x) \phi(t, x)
            + \tilde \phi(t, x) \phi(t, x)
            + \tilde \psi(t, x)\psi(t, x)
        \right\}
    \bigg].
\end{align}
%
If we initialize a particle as a left-mover, $\psi$ at $x_0,t_0$, we find
%
\begin{align}
    \E{\phi(t, x) \tilde \psi(t_0, x_0)}
    =
    0 
    + 
    \parbox{12mm}{
        \centering
        \begin{fmfgraph*}(12,8)
            \setval
            \fmfleft{i}
            \fmfright{o}
            \fmf{plain}{i,c}
            \fmf{wiggly}{c,o}
            \fmfv{l=$\tau$,l.a=90}{c}
        \end{fmfgraph*}
    }
    +
    \parbox{20mm}{
        \centering
        \begin{fmfgraph*}(20,8)
            \setval
            \fmfleft{i}
            \fmfright{o}
            \fmf{plain}{i,c1}
            \fmf{wiggly}{c1,c2}
            \fmf{plain}{c2,c3}
            \fmf{wiggly}{c3,o}
            \fmfv{l=$\tau$,l.a=90}{c1}
            \fmfv{l=$\tau$,l.a=90}{c2}
            \fmfv{l=$\tau$,l.a=90}{c3}
        \end{fmfgraph*}
    }+\dots.
\end{align}
%
See~\cite{garcia-millanRunandtumbleMotionHarmonic2021} for more on this system.




\section{Deans equation}

Another way of getting a field theory for point particles is to derive Dean's equatoin.
In this case, we begin with a Langevin equation for point-particles, $x_i$, which interact via a potential $V()$
%
\begin{align}
    \odv{ x_i(t) }{ t } = - \sum_{j=1}^N \nabla V\left(x_i(t)- x_j(t)\right) + \eta_i(t).
\end{align}
%
Here, $\nabla V(0) = 0$ and $\eta$ is white noise with strength $T$.
We then consider the density function, $\rho(x, t) = \sum_i \delta(x - x_i(t))$.
We leave the derivation to the original publication~\cite{deanLangevinEquationDensity1996}\footnote{Pre-print: \url{https://arxiv.org/abs/cond-mat/9611104}}, but using Itô calculus, we can find
%
\begin{align}
    \partial_t \rho(x, t)
    = \bm \nabla \cdot
    \left[
        T \bm \nabla \rho(x, t) + \rho(x, t) \int \dd y \rho(y, t) \bm \nabla V(x-y) + \sqrt{ 2 T \rho(x, t)} \bm\eta(x, t)
    \right],
\end{align}
%
where $\eta(x,t)$ is unit white noise.

This equation is derived without any approximations, and is thus equivalent with the original Langevin particle equations.
This means that, if the density $\rho$ is initialized as a set of Dirac-deltas, it will remain so, even though there is a diffusion term.
Of course, one may also apply approximations, such as $\rho(x, t) = \bar \rho + \delta \rho$ to get a more familiar Langeivn equation with additive noise, however the `point particle structure' of the time-dependent solution will be lost upon carrying out this approximation.

We may now write down the response field action, 
%
\begin{align}
    A = \int \dd t \dd x \,
    \left\{
        \tilde \rho(\partial_t\rho - D \nabla^2 \rho)
        - D (\nabla \tilde \rho)^2 \rho
    \right\}.
\end{align}
%
The noise term, $D (\nabla \tilde \rho)^2 \rho$, is now a three-point vertex,
%
\begin{align}
    D (\nabla \tilde \rho)^2 \rho
    = 
    \parbox{12mm}{
        \centering
        \begin{fmfgraph*}(12,5)
            \setval
            \fmfleft{i1,i2}
            \fmfright{o}
            \fmf{plain}{i1,c}
            \fmf{plain}{i2,c}
            \fmf{plain,tension=2}{c,o}
        \end{fmfgraph*}
    }
\end{align}
%
Expectation values are then
%
\begin{align}
    \E{\bullet } = \int \D \rho \D \tilde \rho \, \bullet e^{-A}.
\end{align}
%
To incorporate initial conditions in the action, with $n_i$ particles at position $x_i$ and time $t_i$, we add a Dirac-delta term:
%
\begin{align}
    A \rightarrow A + \tilde \rho \sum_i n_i \delta(t + t_i)\delta(x + x_i).
\end{align}
%
Here, we must have $n_i \in \NN$, which again is a sign that the Dean's equation has a particle-nature
To be more specific, we can define a \emph{particle entity signature} $s$.
If the density has the form
%
\begin{align}
    \rho(t, x) &= \sum_i n_i \delta(x - x_i(t)), &
    n_i \in \NN
\end{align}
%
Then, for all sub-domains $\Omega \in \R^d$,
%
\begin{align}
    \int_\Omega \dd x \, \rho(x, t) \in \NN
    \implies 
    \exp \left\{ 2 \pi i \int_\Omega \dd x \, \rho(x, t) \right\} = 1.
\end{align}
%
We therefore define
%
\begin{align}
    s = \E{ 
        \exp \left\{ 2 \pi i \int_\Omega \dd x \, \rho(x, t) \right\} 
    }.
\end{align}
%
This can, when we expand the exponential, be considered a sum of diagrams with $n$ external $\rho$ diagrams:
%
\begin{align}
    s = 
    \sum_n
    \frac{ (2\pi i )^n }{ n! }
    \parbox{15mm}{
        \centering
        \begin{fmfgraph*}(15,10)
            \setval
            \fmfleft{i1,i2,i3,i4}
            \fmfright{o}
            \fmf{plain}{i1,c}
            \fmf{plain}{i2,c}
            \fmf{phantom}{i3,c}
            \fmf{plain}{i4,c}
            \fmf{phantom,tension=4}{c,o}
            \fmfv{l=$\vdots$,l.d=.08w,l.a=22}{i3}
            \fmfv{d.f=empty,d.shape=circle,d.size=5mm,l=$n$,l.d=0}{c}
        \end{fmfgraph*}
    }
\end{align}
%
This is now an observable we can calculate for a given theoryI

Let's consider this in a Doi-Peliti model.
The particle-entity signature is then, using \autoref{eq: moments in 2nd q}, 
%
\begin{align}
    s = 
    \braket{
        \star
        |
            \exp \left\{ z {\sum}_{k \in \Omega} a_k^\dagger a_k\right\}
        |
        \Em(t)
    }\big|_{z = 2 \pi i}.
\end{align}
%
Expanding the exponential and using \autoref{eq: nomral ordering moments}, we get
%
\begin{align}
    \exp \left\{ z {\sum}_{k \in \Omega} a_k^\dagger a_k\right\}
    = 
    \sum_n \frac{ 1 }{ n! } z^n
    \sum_k \left\{ {n \atop k} \right\}
    (a^\dagger_k)^n a^n,
\end{align}
%
so
%
\begin{align}
    s = 
    \E{
        \sum_n \frac{ z^n }{ n! }
        \sum_k \left\{ {n \atop k} \right\} \phi^k
    }
    =
    \E{
        \exp \left\{ \phi (e^{z} - 1) \right\}
    }
    \overset{z=2\pi i}{=}
    1.
\end{align}
%
\todo[inline]{I have no idea how the last eq. works, should be detailed more\dots}

If we go back to Dean's equation, we then we can calculate $s$ diagrammatically.
In fact, it is easier to calculate $\ln s$, as this contains only connected diagrams, as $\ln \E{e^{t X}}$ is the \emph{cumulant generatig function} of $X$, so
%
\begin{align}
    \ln s = \ln 
    \E{
        \exp \left\{ 2 \pi i \int_\Omega \dd x \, \rho(x, t) \right\} 
    }
    = 2 \pi i \ell, \quad \ell \in \Z.
\end{align}
%
We then have
%
\begin{align}
    \parbox{15mm}{
        \centering
        \begin{fmfgraph*}(10,10)
            \setval
            \fmfleft{i1}
            \fmfright{c}
            \fmf{plain}{i1,c}
            \fmfv{d.f=empty,d.shape=circle,d.size=5mm,l=$1$,l.d=0}{c}
        \end{fmfgraph*}
    }
    & = 
    \parbox{10mm}{
        \centering
        \begin{fmfgraph*}(10,10)
            \setval
            \fmfleft{i1}
            \fmfright{c}
            \fmf{plain}{i1,c}
            \fmfdot{c}
        \end{fmfgraph*}
    }\\
    \parbox{15mm}{
        \centering
        \begin{fmfgraph*}(10,6)
            \setval
            \fmfleft{i1,i2}
            \fmfright{c}
            \fmf{plain}{i1,c}
            \fmf{plain}{i2,c}
            \fmfv{d.f=empty,d.shape=circle,d.size=5mm,l=$2$,l.d=0}{c}
        \end{fmfgraph*}
    }
    & = 
    \parbox{10mm}{
        \centering
        \begin{fmfgraph*}(10,6)
            \setval
            \fmfleft{i1,i2}
            \fmfright{o}
            \fmf{plain}{i1,c}
            \fmf{plain}{i2,c}
            \fmf{plain,tension=2}{c,o}
            \fmfdot{o}
        \end{fmfgraph*}
    }\,,
    \quad\text{but not}\quad
    \parbox{10mm}{
        \centering
        \begin{fmfgraph*}(10,4)
            \setval
            \fmfleft{i1,i2}
            \fmfright{o1,o2}
            \fmf{plain}{i1,o1}
            \fmf{plain}{i2,o2}
            \fmfdot{o1}
            \fmfdot{o2}
        \end{fmfgraph*}
    }\\[5mm]
    \parbox{15mm}{
        \centering
        \begin{fmfgraph*}(10,10)
            \setval
            \fmfleft{i1,i2,i3}
            \fmfright{c}
            \fmf{plain}{i1,c}
            \fmf{plain}{i2,c}
            \fmf{plain}{i3,c}
            \fmfv{d.f=empty,d.shape=circle,d.size=5mm,l=$3$,l.d=0}{c}
        \end{fmfgraph*}
    }
    & = 
    \parbox{15mm}{
        \centering
        \begin{fmfgraph*}(10,10)
            \setval
            \fmfleft{i1,i2,i3}
            \fmfright{o}
            \fmf{plain,tension=2}{i1,c1,c}
            \fmf{plain,tension=1}{i3,c}
            \fmf{plain,tension=2}{c,o}
            \fmfdot{o}
            \fmffreeze
            \fmf{plain}{i2,c1}
            \fmfv{l=$1$,l.a=180,l.d=.1w}{i1}
            \fmfv{l=$2$,l.a=180,l.d=.04w}{i2}
            \fmfv{l=$3$,l.a=180,l.d=.1w}{i3}
        \end{fmfgraph*}
    }
    +
    \parbox{15mm}{
        \centering
        \begin{fmfgraph*}(10,10)
            \setval
            \fmfleft{i1,i2,i3}
            \fmfright{o}
            \fmf{plain,tension=2}{i1,c1,c}
            \fmf{plain,tension=1}{i3,c}
            \fmf{plain,tension=2}{c,o}
            \fmfdot{o}
            \fmffreeze
            \fmf{plain}{i2,c1}
            \fmfv{l=$1$,l.a=180,l.d=.1w}{i1}
            \fmfv{l=$3$,l.a=180,l.d=.04w}{i2}
            \fmfv{l=$2$,l.a=180,l.d=.1w}{i3}
        \end{fmfgraph*}
    }
    +
    \parbox{15mm}{
        \centering
        \begin{fmfgraph*}(10,10)
            \setval
            \fmfleft{i1,i2,i3}
            \fmfright{o}
            \fmf{plain,tension=2}{i1,c1,c}
            \fmf{plain,tension=1}{i3,c}
            \fmf{plain,tension=2}{c,o}
            \fmfdot{o}
            \fmffreeze
            \fmf{plain}{i2,c1}
            \fmfv{l=$2$,l.a=180,l.d=.1w}{i1}
            \fmfv{l=$3$,l.a=180,l.d=.04w}{i2}
            \fmfv{l=$1$,l.a=180,l.d=.1w}{i3}
        \end{fmfgraph*}
    }
\end{align}
%
and so on.
We will not show this here, but with this calculation, we indeed find that $s=1$~\cite{botheParticleEntityDoi2023}.

