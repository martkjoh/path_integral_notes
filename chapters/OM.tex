In this section, we will derive the Onsager-Machlup path integral.
This is a formulation of Markovian stochastic processes which allows for describing the conditional probability of the resulting state physical process, given its initial conditions, as a sum over all possible ways, or ``paths'', that could result in that state.


\section{Stochastic processes}

A stochastic process is $x(t)$ is a function of $t \in \Te$, where the only requirement on $\Te$ is that it is \emph{total ordered}.
This means that, for any two elements $t_1, t_2 \in \Te$ and $t_1 \neq t_2$, then $t_1 < t_2$ or $t_2 < t_1$.
The most common example of $\Te$ is time, so $x(t)$ describes the evolution of $x$ thought time.
However, $\Te$ might also be the links in a polymer, \dots \todo{examples}
Furthermore, $\Te$ can be discrete or continuous.
We will begin by considering discrete time steps with a length of $\Delta t$, so
%
\begin{align}
    \Te = \Delta t \Z.
\end{align}
%

A \emph{Markov process} is a stochastic process with ``no memory''.
This means that, if we have $n$ steps of the process $x(t_1), x(t_2), \dots x(t_n)$, then the conditional probability of $x(t_n)$, given $x(t_1), \dots x(t_{n-1})$, only depends on the last step $x(t_{n-1})$.
Using the common notation for conditional probabilities, where $P(A|B)$ means ``the probability of $A$ given $B$'', this can be stated as
%
\begin{align}
    P(x_n |x_{n-1}, x_{n-2} \dots x_1) = P(x_n | x_{n-1}).
\end{align}
%
Here, we use the shorthand $x_i = x(t_i)$.

As far as we know, the underlying laws of physics are Markovic, so the question of 

\note{More details on what markovian means}

This does not imply statistical independence, so $P(x_{n}, x_{n-1})\neq P(x_n)P(x_n)$


A Markovian process has the property
%
\begin{align}
    P(x_1, x_2, x_3) = P(x_3|x_2)P(x_2|x_1)P(x_1).
\end{align}
%
From this, we may derive the Chapman-Kolmogorov equation,\todo{detail?}
%
\begin{align}\label{eq: chapman kolmogorov}
    P(x_3|x_1) = \sum_{x_2} P(x_3|x_2) P(x_2|x_1).
\end{align}
%
Here, the sum is over all possible values of $x_2$.
\note{What is x? Something like: we consider $x$ discrete, $x \in \Delta x \Z $. }

Repeatedly applying this gives the conditional probability of two steps $x_1$ and $x_{n+1}$ arbitrarily far removed, 
%
\begin{align}\label{eq: cond prob markov x0 given xn}
    P(x_{n+1}|x_0) 
    = \sum_{\{ x_1, \dots x_n \}}
    P(x_{n+1}|x_n) P(x_n| x_{n-1})\dots P(x_2|x_1).
\end{align}
%
We see that this already begins to resemble something like a sum over all possible paths.

\note{more text}

\note{Illustration}




\section{Gaussian process and the Ornstein-Uhlenbeck process}

The most important class of stochastic processes, as it is pretty much the only one we can solve, is Gaussian processes.
These have the form
%
\begin{align}
    P(x_{n_1}|x_n)
    = 
    \frac{ 1 }{ \sqrt{ 2 \pi \det \Sigma } }
    \exp \left\{ - (x_{n+1}  - \bar x_{n+1}) \Sigma^{-1} (x_{n+1}  - \bar x_{n+1}) \right\}.
\end{align}
%
Here, $\bar x_{n+1}$ is the expected value of $x_n$.
As we consider Markovian processes, this is a function of only the previous step, $x_n$
$\Sigma$ is the covariance of the process.

\todo{Why matrix notation?}

A specific example is the Ornstein-Uhlenbeck process,
%
\begin{align}
    \odv{ x(t) }{ t } = - \mu x(t) + \eta(t).
\end{align}
%

\note{More about this process \dots}

This has the conditional probability\todo{where does $\Delta x$ come in here?}
%
\begin{align}
    P(x_{n + 1}| x_n) 
    = \sqrt{ \frac{ \mu }{ 4 \pi D \Delta t } }
    \exp \left\{ 
    \frac{ \left(x_{n + 1} - x_n + \mu x_n \Delta t\right)^2 }{ 4 D \Delta t } 
    \right\}
    = \sqrt{ \frac{ \mu }{ 4 \pi D \Delta t } }
    \exp \left\{ 
    \frac{ \Delta t }{ 4 D }  \left(\dot x_{n + 1} + \mu x_n\right)^2
    \right\},
\end{align}
%
where we have introduced the discrete time-derivative, $\dot x_{n+1} = (x_{n + 1} - x_n) / \Delta t$.

We will now take two different limits, which will yield the path integral.
This means we will have to take the continuum limit.

\note{Is this the right way to do it?}
To do this, we must consider probability \emph{densities}, $\Pe(x) = P(x) / \Delta x $.
The densities are ``probability of the value $x$ per $\Delta x$''.
This allows us to go from sums to integrals,
%
\begin{align}
    \sum_{x} P(x) = \sum_{x} \Delta x \Pe(x) \underset{\Delta x \rightarrow 0}{\longrightarrow} \int \dd x \, \Pe(x).
\end{align}
%

\note{Define $\D x(t)$}

If we insert this into \autoref{eq: cond prob markov x0 given xn}, we get
%
\begin{align}
    P(x_{n + 1} | x_1) 
    = \int \D x(t) \,
    \exp \left\{ 
        - \frac{ 1 }{ 4D } 
        \int\limits_{t_1}^{t_{n+1}} \dd t \left[\dot x(t) + \mu x(t)\right]^2
        \right\}
    \equiv
    \int \D x(t) \, \Phi[x(t)].
\end{align}
%
$\Phi$ is the Onsager-Machlup \emph{functional}.
It is a functional, and not a function, as it takes in a function $x(t)$, and gives back a number.

\section{Master equation}

We will here consider stochastic processes using a different formalism, the master equation.
If we consider a random process $N(t)$,\todo{Why not $x$?} and Taylor-expand the conditional probability of $N'$ at $t + \Delta t$ given $N$ at $t$ in small $\Delta t$, we get
%
\begin{align}\label{eq: expansion cond prob}
    P(N'(t + \Delta t)| N(t)) = 
    \underbrace{P(N'(t)|N(t))}_{\delta_{N'N}} + \Delta t 
    \underbrace{\pdv{P(N'(t')|N(t))}{t'}\bigg|_{t'=t}}_{W_t(N'|N)} + \Oh(\Delta t).
\end{align}
%
Here, we define the transition rates $W_t(N'|N)$.
This can be considered as a matrix, 
%
\begin{align}
    W_t(N'|N) = 
    \begin{pmatrix}
        W_t(1|1) & W_t(1|2)& W_t(1|3) &\cdots\\
        W_t(2|1) & W_t(2|2)& W_t(2|3) &\cdots \\
        \vdots & \vdots & \ddots & \cdots
    \end{pmatrix},
\end{align}
%
where the entries are the rate at which state $N'$ transitions to state $N$.
If we consider $N' = N$, then we can, by the principle of total probability (the probability of all possible outcomes add up to one), we can write
%
\begin{align}
    P(N(t + \Delta t)| N(t)) = 1 + \Delta t W_t(N|N) 
    = 1 - \sum_{N' \neq N}P(N'(t + \Delta t)| N(t))
    = 1 - \sum_{N' \neq N}\Delta t W_t(N',N),
\end{align}
%
or
%
\begin{align}\label{eq: rate cons condition}
    W_t(N|N) = - \sum_{N'\neq N}W_t(N'|N)
\end{align}
%
This insures the conservation of probability.
If we consider the matrix form of the rates, this means the diagonals of $W_t(N'|N)$ are given by minus the sum of each column,
%
\begin{align}
    W_t(N'|N) = 
    \begin{pmatrix}
        - \sum_{N\neq 1} W_t(N|1) & W_t(1|2)& W_t(1|3) &\cdots\\
        W_t(2|1) & - \sum_{N\neq 2} W_t(N|2)& W_t(2|3) &\cdots \\
        \vdots & \vdots & \ddots & \cdots
    \end{pmatrix},
\end{align}
%
\todo{is this right?}
If we expand the Chapman-Kolmogorov, \autoref{eq: chapman kolmogorov},
%
\begin{align}
    P(N(t_3) |N(t_1))
    & =
    \sum_{N(t_2)} 
    \left[
        \delta_{N(t_3)N(t_2)}
        + (t_3 - t_2) W_{t_2}(N(t_3)|N(t_2))
    \right]
    P(N(t_2)|N(t_1))
    \\
    & =
    (t_3 - t_2)\sum_{N(t_2)\neq N(t_3)} 
        W_{t_2}(N(t_3)|N(t_2)) P(N(t_2)|N(t_1))\\
    & \quad 
    + \left[ 1 + (t_3 - t_2) W_{t_2}(N(t_3)|N(t_2)=N(t_3))  \right] P(N(t_2)=N(t_3)|N(t_1))
\end{align}
%
If we apply \autoref{eq: rate cons condition}, and define $\Delta t = t_3 - t_2$, we can rearrange this to read
%
\begin{align}
    &\frac{P(N(t_3)|N(t_1)) - P(N(t_2)=N(t_3)|N(t_1))}{\Delta t}\\
    &=
    \sum_{N(t_2) \neq N(t_3)}
    \left[
        W_{t_2}(N(t_3)|N(t_2))P(N(t_2)|N(t_1))
        - W_{t_2}(N(t_2)|N(t_2)=N(t_3))P(N(t_2)=N(t_3)|N(t_1))
    \right]
\end{align}
%
The top line of this equation is the difference in probability of finding the system in the state $N(t_3)$ at time $t_3$ and $N(t_2)$, so in the limit $\Delta t \rightarrow 0$ becomes a derivative.
On the bottom, we have the diffenrence between the \emph{gain} rate and the \emph{loss} rate in to and out of state $N(t_3)$
We take the limit, which gives the \emph{master equation},
%
\begin{align}
    \partial_t P(N(t)|N_0) =
    \sum_{N'\neq N} \left[
        W_t(N(t)|N'(t))P(N'(t)|N_0)
        - 
        W_t(N'(t)|N(t))P(N(t)|N_0)
    \right].
\end{align}
%

The correspondence between the master-equation and the Onsager-Machlup functional is analogous to the relationship between the Schrödinger equation and the Feynamn path integral in QM.

\note{more text on this}



\section{Gaussian integrals}

The most important integrals we will meet are \emph{Gaussian integrals}.
The simplest Gaussian integral, which will be the basis for this chapter, is
%
\begin{align}\label{eq: gaussian integral}
    \int\limits_{-\infty}^\infty \dd x \, e^{-\frac{1}{2} x^2} = \sqrt{ 2 \pi }.
\end{align}
%
The reason we can solve Gaussian processes, as we mentioned earlier, is that we can solve Gaussian integrals.
With this simple result, we can solve path integrals over spaces of functions.
Consider\todo{what }
%
\begin{align}
    Z = \int \D \phi
    \exp \left\{ 
        - \frac{1}{2} \int \dd t_1 \dd t_2 \, 
        \sum_{ab}
        \phi_a(t_1) A_{ab}(t_1, t_2) \phi_b(t_2)
    \right\}.
\end{align}
%
Here, $\phi(t_1)$ is a vector function with $d$ components, $\phi_a(t) = (\phi_1(t), \phi_2(t)\dots)^T$, and $A_{ab}(t_1, t_2)$ is a matrix-function.
In the case of Markov-processes, $A(t_1, t_2)$ will be \emph{time-local}.
This means $A(t_1, t_2) = A(t_1) \delta(t_1 - t_2)$
For the path integral $Z$ to be well-defined, we must discretize $t_n = n \Delta t$, with $n \in \Z$
This gives us an extra set of indices, $\phi_a(t_i) = \phi_{a,i}$.
However, we can deal with this by ``unpacking'' the two-dimensional structure of $\phi_{a,i}$,
%
\begin{align}
    \phi_\alpha = 
    \left[\phi_1, ... \phi_N\right]
    \left[
        \phi_{1, 1}, \phi_{2, 1}\dots\phi_{d,1}, \phi_{1, 2}\dots\phi_{d, n}
    \right],
\end{align}
%
where $N = d \times n$.
This is analogous to ``flattening'' and array in computer programming.
We also apply this to $A$.
The discretized dirac-delta is $\delta(t_1 - t_2) \sim \frac{1}{\Delta t} \delta_{t_1, t_2}$, so factoring out this $\Delta t$ we get 
%
\begin{align}
    A_{ab}(t_1,t_2)
    \sim \frac{1}{\Delta t} A_{ab,ij}
    \rightarrow \frac{1}{\Delta t} A_{\alpha \beta},
\end{align}
%
The path-internal then discretizes as 
%
\begin{align}
    Z = \int \left( \prod_{\alpha=1}^N \dd \phi_\alpha \right) \, 
    \exp \left\{ 
        \frac{1}{2} \Delta t \sum_{\alpha \beta} 
        \phi_\alpha A_{\alpha \beta} \phi_\beta
    \right\}.
\end{align}
%
We now assume $A$ is diagonalizable, so we can write it as $A = G \Lambda G^{-1} $, where $\Lambda = \mathrm{\lambda_1, ... \lambda_N}$ is diagonal, with $\lambda_\alpha$ are the eigenvalues of $A$.
We can therefore perform a change of variables, $z = G^{-1}\phi$, which simplifies the integral\todo{what about jacobian}
%
\begin{align}
    Z = \int \left( \prod_{\alpha=1}^N \dd z_\alpha \right)
    \exp \left\{ \frac{1}{2}\Delta t \sum_{\alpha} \lambda_\alpha z_\alpha^2 \right\}
    =
    \sqrt{ \det 2 \pi (\Delta t A)^{-1} }.
\end{align}
%
Here, we factored the exponential, used the simple Gaussian integral \autoref{eq: gaussian integral}, and used the fact that the product of the eigenvalues is the determinant.
