In this chapter, we will consider Langevin equations with the form
%
\begin{align}\label{eq: langevin for response}
    \odv{  }{ t } \phi(t)
    \overset{\alpha}{=}
    f(\phi(t)) + h(t) + \zeta(t),
\end{align}
%
where
%
\begin{align}
    \E{\zeta(t) \zeta(t')} = 2 T \delta(t - t').
\end{align}
%
This lecture is based on \footnote{\cite{hertzPathIntegralMethods2016}---\fullcite{hertzPathIntegralMethods2016}}, so we refer to these notes for the derivation---they are free to download.
This contains the calculations of correlation functions and response function,
%
\begin{align}
    C(t - t') &= \E{\phi(t)\phi(t')} \big|_{h=0}, &
    R(t - t') &= \fdv{ \E{\phi(t)} }{ h(t') } \bigg|_{h=0}
    =
    \frac{1}{2 T}
    \E{\phi(t)\zeta(t')} \big|_{h=0}.
\end{align}
%

As shown by \cite{hertzPathIntegralMethods2016}, the discretized version of \autoref{eq: langevin for response} is
%
\begin{align}
    \label{eq: discretized langevin for response}
    \phi_{n + 1} = \phi_n + \Delta t 
    \left\{
        (1 - \alpha) (f_n + h_n) + \alpha (f_{n+1} + h_{n+1})
    \right\} + \zeta_n
\end{align}
%
Here, $f_n = f(x(t_n))$ and $h_n = h(t_n)$, while
%
\begin{align}
    \zeta_n = \int_{t_n}^{t_{n+1}} \dd \tau \, \zeta(t).
\end{align}
%
With this, the \emph{generating functional} is
%
\begin{align}
    Z[\psi]
    \equiv
    \E{\exp \left\{ i \sum_n \psi_n \phi_n \right\}}
    =\int \dd \phi \, 
    \exp \left\{ - \Delta t \alpha \sum_{n=1}^M f'_n + i \sum_n \psi_n \phi_n  \right\}
    \times
    \frac{ 1 }{ \sqrt{ 4 \pi \Delta t T } }
    \exp \left\{ - \frac{ 1 }{ 4 T \Delta t } \sum_n \zeta_n^2(\phi) \right\}
\end{align}
%
where $\zeta_n(\phi)$ is given by \autoref{eq: discretized langevin for response}.
This is essentially the Onsager-Machlup functional~\autoref{eq:om_func}.

\note{%
    Add summary of renormalizing $R$.
}


\section{Supersymmetry}

We may extend the response field formalism by introducing \emph{Grassmann} variables $\xi_i$, which obey a certain symmetry: super-symmetry.
We briefly summarize the results here, the full details can be found in~\cite{hertzPathIntegralMethods2016}.

\subsection{Grassmann numbers}

We can define Grassmann numbers $\xi_i$ by the properties 
%
\begin{align}
    \xi_i \xi_j &= - \xi_j \xi_i, \\
    \xi_i (\xi_j \xi_k) &= ( \xi_j \xi_i) \xi_k,
\end{align}
%
\emph{anti-commuation} and \emph{associativity}.
As a consequence, $\xi^n = 0$, for $n > 1$.
Any function of a single Grassmann number therefore takes the form
%
\begin{align}
    f(\xi) = a + b \xi,
\end{align}
%
for constants $a$ and $b$, and in particular
%
\begin{align}
    e^\xi = 1 + \xi.
\end{align}
%
For functions of two variables,
%
\begin{align}
    f(\xi_1,\xi_2) = c_0 + c_1 \xi_1 + c_2 \xi_2 + c_{12} \xi_1 \xi_2,
\end{align}
%
and so on.


We may define the derivative by
%
\begin{align}
    \odv{  }{ \xi_i } \xi_j = \delta_{ij}.
\end{align}
%
This acts on the leftmost Grassmann number so
%
\begin{align}
    \odv{  }{ \xi_1 } f(\xi_1, \xi_1)
    &= c_1 + c_{12} \xi_2,&
    \odv{  }{ \xi_2 } f(\xi_1, \xi_1)
    &= c_2 - c_{12} \xi_1.
\end{align}
%
The derivative operators anti-commute.
Likewise, the integral is defined by
%
\begin{align}
    \int \dd \xi 1 &= 0,&
    \int \dd \xi \, \xi &= 1,
\end{align}
%
and
%
\begin{align}
    \int \dd \xi_1 \dd \xi_2 \xi_2 \xi_1
    = - \int \dd \xi_1 \dd \xi_2 \xi_1 \xi_2
    = 1.
\end{align}
%
With this, we may write the determinant of a matrix $A_{ij}$ as
%
\begin{align}
    \int \D \xi \D \bar \xi \exp \left\{ {\sum}_{ij} \bar \xi_i A_{ij} \xi_j \right\}
    = \det A,
\end{align}
%
where $\D\xi \D \bar \xi = \prod_{i} \dd \bar \xi_i \dd \xi$.
Furthermore,
%
\begin{align}
    \E{\xi_i \bar \xi_j}
    =
    \int \D \xi \D \bar \xi \exp \left\{ {\sum}_{ij} \bar \xi_i A_{ij} \xi_j \right\}
    = 
    A_{ij}^{-1}.
\end{align}
%

With this, we can write the determinant from integrating out the noise, $\zeta$, as integral over Grassmann numbers:
%
\begin{align}
    J[\phi]
    =
    \int \D \bar \xi \D \xi
    \,
    \exp \left\{ \sum_n
    \left[
        \bar\xi_n \left(1 - \alpha \Delta t f'_{n+1}\right)\xi_{n+1}
        - \bar\xi_n \left(1 - (\alpha - 1) \Delta t f'_{n}\right)\xi_{n}
    \right]
     \right\}.
\end{align}
%
This is then added to the action, and expectation values are now
%
\begin{align}
    \E{\bullet}
    = 
    \int \D\tilde \phi \D \phi \D \bar \xi \D \xi \, \bullet e^{-A},
\end{align}
%
where
%
\begin{align}
    &A = \\\nonumber
    &\sum_n
    \left\{
        T \Delta t \tilde \phi_n^2
        - i \tilde \phi_n \left(
            -\phi_{n+1} + \phi_n + \delta t\left[(1 - \alpha)f_n + f_{n+1}\right]
            \right)
            -
            \bar\xi_n \left[1 - \alpha \Delta t f'_{n+1}\right]\xi_{n+1}
            + \bar\xi_n \left[1 - (\alpha - 1) \Delta t f'_{n}\right]\xi_{n}
    \right\}.
\end{align}
%